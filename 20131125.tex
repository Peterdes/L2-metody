\documentclass[12pt]{article}
\usepackage{polski}
\usepackage[utf8]{inputenc}
\usepackage[T1]{fontenc}
\usepackage{amsmath}
\usepackage{amsfonts}
\usepackage{fancyhdr}
\usepackage{lastpage}
\usepackage{multirow}
\usepackage{amssymb}
\usepackage{amsthm}
\usepackage{textcomp}
\usepackage{mathabx}
\usepackage{mathtools}
\usepackage{bbm}
\frenchspacing
\usepackage{fullpage}
\setlength{\headsep}{30pt}
\setlength{\headheight}{12pt}
%\setlength{\voffset}{-30pt}
%\setlength{\textheight}{730pt}
\pagestyle{myheadings}

\usepackage{tikz}
%\usepackage{tikz-cd}
\usetikzlibrary{arrows}

\newcommand{\bigslant}[2]{{\left.\raisebox{.2em}{$#1$}\middle/\raisebox{-.2em}{$#2$}\right.}}
\newcommand{\im}{{\mathrm{im}\,}}

\newcommand{\mf}[1]{{\mathfrak{#1}}}
\newcommand{\mb}[1]{{\mathbb{#1}}}
\newcommand{\mc}[1]{{\mathcal{#1}}}
\newcommand{\mr}[1]{{\mathrm{#1}}}

\newcommand{\R}{{\mathbb{R}}}
\newcommand{\Z}{{\mathbb{Z}}}
\newcommand{\N}{{\mathcal{N}}}
\newcommand{\tr}{{\mathrm{tr}}}
\renewcommand{\H}{{\mathcal{H}}}
\newcommand{\1}{{\mathbbm{1}}}
\newcommand{\Hom}{{\mathrm{Hom}}}
\newcommand{\Id}{{\mathrm{Id}}}
\newcommand{\simp}[1]{{\Delta^{#1}}}
\newcommand{\osimp}[1]{{\mathring{\Delta}^{#1}}}

\newcounter{punkt}

\theoremstyle{plain}
\newtheorem{twierdzenie}[punkt]{Twierdzenie}
\newtheorem{twierdzeniebd}[punkt]{Twierdzenie (bez dowodu)}
\newtheorem{lemat}[punkt]{Lemat}
\newtheorem{stwierdzenie}[punkt]{Stwierdzenie}
\newtheorem{hipoteza}[punkt]{Hipoteza}

\theoremstyle{definition}
\newtheorem{definicja}[punkt]{Definicja}
\newtheorem{fakt}[punkt]{Fakt}
\newtheorem{wniosek}[punkt]{Wniosek}

\theoremstyle{remark}
\newtheorem{uwaga}[punkt]{Uwaga}
\newtheorem{przyklad}[punkt]{Przykład}
\newtheorem{cwiczenie}[punkt]{Ćwiczenie}



\markright{Piotr Suwara\hfill $\ell_2$-metody w~topologii: 25 listopada 2013\hfill}
 
\begin{document}

Pomijam trochę F\o{}lnerowskich rzeczy.

\begin{definicja}
	Istnieje kanoniczne przekształcenie
	$\mr{can}^i: \bar{H}^i(Y) \to H^i(Y, \R)$,
	$\varphi \mapsto \varphi(\cdot)(e)$.
\end{definicja}

\begin{lemat}[Cheegar - Gromov]
	$Y$ spójny, wolny, kozwarty $G$-kompleks, 
	$G$ nieskończona grupa średniowalna,
	wówczas $\mr{can}^i : \bar{H}^i(Y) \to H^i(Y, \R)$
	jest włożeniem.
\end{lemat}

\begin{uwaga}
	Żeby mieć liczby Bettiego chcemy, aby $G$ była 
	skończenie prezentowalna.
	Dla powyższych założeń i~tak $G$ musi być skończenie generowana
	(lemat Milnora-Schwartza).
\end{uwaga}

\begin{hipoteza}[Gromov]
	Zredukowane $\ell_p$-kohomologie grupy średniowalnej znikają.
\end{hipoteza}

\begin{wniosek}
	$X$ skończony, $\pi_1(X) = G$ średniowalna,
	wówczas $\beta_1(X) = \beta_1(G) = 0$.
\end{wniosek}

\begin{stwierdzenie}
	$X$ spójny kompleks o~skończonym $2$-szkielecie.
	Wówczas $\beta_1(X) = \beta_1(\pi(X))$.
\end{stwierdzenie}

\begin{wniosek}
	$\mb{F}_n$ nie jest średniowalna dla $n \geq 2$.
\end{wniosek}

\begin{twierdzenie}[Cheegar-Gromov]
	$G$ skończenie prezentowalna, $|G|= \infty$, średniowalna,
	wtedy $\beta_1(G) = 0$. Jeśli $G$ jest typu $F_m$,
	to $\beta_i(G) = 0$ dla $i \leq m-1$.
\end{twierdzenie}

\begin{wniosek}
	$G$ nieskończona średniowalna o~skończonej $K(G, 1)$,
	wówczas $\chi(G) = 0$.
\end{wniosek}

{\bf Defekt grupy}

Załóżmy, że $G$ posiada prezentację o~$g$ generatorach i~$r$ relacjach.

\begin{definicja}
	$\mr{def}(G) = \max \{ g - r \}$ -- maksimum po skończonych
	prezentacjach.
\end{definicja}

\begin{fakt}
	$\mr{def}(G) \leq b_1(G) - b_2(G)$
	
	$\mr{def}(G) = 1 - \beta_0(G) + \beta_1(G) - \beta_2(K(G, 1)^{(2)})$
\end{fakt}

\begin{wniosek}
	$\mr{def}(G) \leq 1 + \beta_1(G)$ dla $G$ skończenie prezentowalnej.
\end{wniosek}














\end{document}
