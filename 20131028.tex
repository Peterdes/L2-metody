\documentclass[12pt]{article}
\usepackage{polski}
\usepackage[utf8]{inputenc}
\usepackage[T1]{fontenc}
\usepackage{amsmath}
\usepackage{amsfonts}
\usepackage{fancyhdr}
\usepackage{lastpage}
\usepackage{multirow}
\usepackage{amssymb}
\usepackage{amsthm}
\usepackage{textcomp}
\usepackage{mathabx}
\usepackage{mathtools}
\usepackage{bbm}
\frenchspacing
\usepackage{fullpage}
\setlength{\headsep}{30pt}
\setlength{\headheight}{12pt}
%\setlength{\voffset}{-30pt}
%\setlength{\textheight}{730pt}
\pagestyle{myheadings}

\usepackage{tikz}
%\usepackage{tikz-cd}
\usetikzlibrary{arrows}

\newcommand{\bigslant}[2]{{\left.\raisebox{.2em}{$#1$}\middle/\raisebox{-.2em}{$#2$}\right.}}
\newcommand{\im}{{\mathrm{im}\,}}

\newcommand{\mf}[1]{{\mathfrak{#1}}}
\newcommand{\mb}[1]{{\mathbb{#1}}}
\newcommand{\mc}[1]{{\mathcal{#1}}}
\newcommand{\mr}[1]{{\mathrm{#1}}}

\newcommand{\R}{{\mathbb{R}}}
\newcommand{\Z}{{\mathbb{Z}}}
\newcommand{\N}{{\mathcal{N}}}
\newcommand{\tr}{{\mathrm{tr}}}
\renewcommand{\H}{{\mathcal{H}}}
\newcommand{\1}{{\mathbbm{1}}}
\newcommand{\Hom}{{\mathrm{Hom}}}
\newcommand{\Id}{{\mathrm{Id}}}
\newcommand{\simp}[1]{{\Delta^{#1}}}
\newcommand{\osimp}[1]{{\mathring{\Delta}^{#1}}}

\newcounter{punkt}

\theoremstyle{plain}
\newtheorem{twierdzenie}[punkt]{Twierdzenie}
\newtheorem{twierdzeniebd}[punkt]{Twierdzenie (bez dowodu)}
\newtheorem{lemat}[punkt]{Lemat}
\newtheorem{stwierdzenie}[punkt]{Stwierdzenie}
\newtheorem{hipoteza}[punkt]{Hipoteza}

\theoremstyle{definition}
\newtheorem{definicja}[punkt]{Definicja}
\newtheorem{fakt}[punkt]{Fakt}
\newtheorem{wniosek}[punkt]{Wniosek}

\theoremstyle{remark}
\newtheorem{uwaga}[punkt]{Uwaga}
\newtheorem{przyklad}[punkt]{Przykład}
\newtheorem{cwiczenie}[punkt]{Ćwiczenie}



\markright{Piotr Suwara\hfill $\ell_2$-metody w~topologii: 28 października 2013\hfill}
 
\begin{document}

{\bf Wymiar Murray'a - von Neumanna}

\begin{uwaga}
	Chcemy zdefiniować $\dim_G$ na Hilbertowskich $G$-modułach t, że:
	\begin{itemize}
		\item $\dim_G M \geq 0$
		\item $\dim_G M  = 0 \iff M = 0$
		\item $\dim_G M = \dim_G N$ jeśli $M \simeq N$
		\item $\dim_G M \oplus N = \dim_G M + \dim_G N$
		\item $\dim_G M \leq \dim_G N$ jeśli $M \leq N$
		\item $\dim_G \ell_2(G) = 1$
		\item $\dim_G M = \frac{1}{|G|} \dim_\R M$ dla $G$ -- skończonej
		\item $\dim_G M = \frac{1}{[G:S]} \dim_G M$ jeśli $S \leq G$
		podgrupa skończonego indeksu
	\end{itemize}

\end{uwaga}

\begin{definicja}[algebra von Neumanna]
	\emph{Algebrą von Neumanna grupy $G$} nazywamy algebrę $\N(G)$ 
	ograniczonych (lewo-) $G$-ekwiwariantnych operatorów 
	${T: \ell_2(G) \to \ell_2(G)}$ 
	(takich, że $\forall_{g \in G} {T \lambda_g = \lambda_g T}$).
	
	$\ell_2(G)$ jest bimodułem nad $\R G$ (przez prawe i~lewe działanie).
\end{definicja}

\begin{uwaga}
	$\R G \subseteq \N(G)$.
\end{uwaga}

\begin{stwierdzenie}
	$\phi \in \N(G) \implies \phi^\ast \in \N(G)$
\end{stwierdzenie}

\begin{wniosek}
	$\N(G)$ jest $\mb{C}^\ast$-algebrą.
\end{wniosek}

\begin{uwaga}
	Na $\R G$, $\ast: \sum r(g) g \mapsto \sum r(g) g^{-1}$.
\end{uwaga}

\begin{uwaga}
	Elementy $\R G$ jako macierze $|G|$ na $|G|$ są stałe na przekątnych
	$\{ (g,h): g^{-1} h = \gamma \}$.
\end{uwaga}

\begin{definicja}[ślad]
	$\varphi \in \N(G)$, to $\tr_G( \varphi) 
	= \langle \varphi(\1), \1 \rangle$
\end{definicja}

\begin{uwaga}
	Na $\R G$, $\varphi = \sum r(g) g$, $\tr_G(\varphi) = r(e)$.
\end{uwaga}

\begin{uwaga}
	Jeśli $G$ skończona, to $\R G = \ell_2(G) = \N(G)$,
	$\varphi$ odpowiada macierz $M_\varphi$ stała na przekątnych,
	$\tr_G(\varphi) = \frac 1 {|G|} \tr(M_\varphi)$.
\end{uwaga}

\begin{stwierdzenie}
	$\tr_G(\varphi) = \tr_G(\varphi^\ast)$
\end{stwierdzenie}

\begin{lemat}
	$\R$-liniowe przekształcenie ${\theta: \N(G) \to \ell_2(G)}$,
	${\theta(\varphi) = \varphi(\1)}$ jest włożeniem
	oraz $\theta(\varphi^\ast) = \overline{\varphi(\1)}$,
	gdzie $\bar{f}(x) = f(x^{-1})$.
\end{lemat}

\begin{wniosek}
	$\tr_G$ jest śladem na $\N(G)$, 
	czyli $\tr_G(\varphi \circ \psi) = \tr_G(\psi \circ \varphi)$.
\end{wniosek}

\begin{cwiczenie}
	Znaleźć opis $\N(G)$ oraz $\tr_G$ dla $G= \Z^n$.
\end{cwiczenie}

\begin{definicja}
	Niech $M_n(\N(G))$ algebra ograniczonych (lewo-) 
	$G$-ekwiwariantnych operatorów na $\ell_2(G)^n$.
	
	Operator $F \in M_n(\N(G))$ jest wyznaczony przez macierz
	$[F_{ij}]$, gdzie $F_{ij} \in \N(G)$ i~spełnia
	$F(a_1, \ldots, a_n) = (\sum F_{1k} a_k, \ldots, \sum F_{nk} a_k)
	\in \ell_2(G)^n$.
\end{definicja}

\begin{uwaga}
	Mamy $(F^\ast)_{ij} = F_{ji}^\ast$.
\end{uwaga}

\begin{definicja}[ślad]
	$\tr_G(F) = \sum_{i=1}^n \tr_G(F_{ii})$
\end{definicja}

\begin{lemat}
	Jeśli $F = F^\ast \in M_n(\N(G))$, $F^2 = F$,
	to $\tr_G(F) = \sum_{i,j} \| F_{ij}(\1) \|^2$.
\end{lemat}

\begin{wniosek}
	$F \in M_n(\N(G))$ samosprzężony idempotentny,
	to $\tr_G F \geq 0$ oraz ${\tr_G F = 0 \implies F = 0}$.
\end{wniosek}

\begin{uwaga}
	To samo jest prawdą dla każdego $F$ idempotentnego,
	bo wtedy dla $\pi$ rzutu ortogonalnego na $\im F$
	mamy $\tr_G \pi = \tr_G F$.
\end{uwaga}

\begin{stwierdzenie}
	Jeśli $P \in \N(G)$ idempotentny, to 
	$\tr_G(P) + \tr_G(\Id-P) = 1$, czyli
	$\tr_G P \leq 1$ oraz $\tr_G(P) = 1 \iff P = \Id$.
\end{stwierdzenie}

\begin{uwaga}
	Jeśli $P \in \Z G$ idempotentny, to $\tr_G P = 0$ lub $\tr_G P = 1$.
\end{uwaga}

\begin{wniosek}
	Jedynymi idempotentami w~$\Z G$ są $0$ oraz $1$.
\end{wniosek}

\begin{hipoteza}[Kaplonsky]
	$G$ beztorsyjna, to $0$ i~$1$ są jedynymi idempotentami w~$\R G$.
\end{hipoteza}

\begin{fakt}
	$V \subseteq \ell_2(G)^n$ -- $G$-niezmiennicza domknięta 
	podprzestrzeń, $\pi_V$ rzut ortogonalny z~$\ell_2(G)$ na $V$.
	
	Wtedy $\pi_V \in M_n(\N(G))$.
\end{fakt}

\begin{definicja}[wymiar von Neumanna]
	$V$ domknięta podprzestrzeń $\ell_2(G)^n$, to 
	${\dim_G V = \tr_G \pi_V}$
\end{definicja}

\begin{stwierdzenie}
	$\pi_B$ idempotent, czyli $\dim_G V \geq 0$ 
	oraz $\dim_G V = 0 \implies V = 0$.
\end{stwierdzenie}

\begin{definicja}[wymiar von Neumanna]
	Niech $M$ -- $G$-moduł Hilbertowski, ustalmy $G$-ekwiwariatną
	izometrię $\alpha: M \to V$, na $V \subseteq \ell_2(G)^n$
	domkniętą podprzestrzeń.
	
	Wtedy $\dim_G M = \dim_G V$.
\end{definicja}

\begin{uwaga}
	Ta definicja nie zależy od wyboru $\alpha$.
\end{uwaga}

\begin{wniosek}
	$\dim_G M \geq 0$
	
	$\dim_G M = 0 \iff M = 0$
\end{wniosek}

\begin{wniosek}
	$\dim_G(M \oplus N) = \dim_G M + \dim_G N$
\end{wniosek}

\begin{wniosek}
	$\dim_G M = \frac 1 {[G:S]} \dim_S M$ dla $S \subseteq G$,
	$[G:S] < \infty$.
\end{wniosek}

\begin{definicja}
	Kompleks łańcuchowy $G$-modułów Hilbertowskich
	${V_\ast: \ldots \to V_{i+1} \to V_i \to V_{i-1} \to \ldots}$
	nazywamy $\ell_2(G)$-kompleksem łańcuchowym, jeśli
	dla każdego $i$, $d_i:V_i \to V_{i-1}$ jest
	ograniczonym $G$-ekwiwariantnym operatorem.
	
	Wówczas zredukowane kohomologie $V_\ast$ są $G$-modułami 
	Hilbertowskimi $\bar{H}_i(V_\ast) 
	= \bigslant{\ker d_i}{\im d_{i+1}}$,
	
	zaś $V_\ast$ nazywa się słabo-dokładny, jeśli 
	$\forall_i \bar{H}_i(V_\ast) = 0$.
\end{definicja}

\begin{definicja}
	$V_\ast, W_\ast$ -- $\ell_2(G)$-kompleksy łańcuchowe.
	\begin{enumerate}
		\item \emph{Morfizmem} $\phi_\ast:V_\ast \to W_\ast$
		nazywamy morfizm kompleksów łańcuchowych
		${\{\phi_i : V_i \to W_i\}}$,
		w~którym operatory są $G$-ekwiwariantne i~ograniczone.
		
		\item Dwa morfizmy $\phi_\ast, \psi_\ast:V_\ast \to W_\ast$
		są \emph{$\ell_2(G)$-homotopijne}, jeśli są homotopijne
		łańcuchowo przez homotopię łańcuchową
		składającą się z~operatorów ograniczonych
		$G$-ekwiwariantnych.
		
		\item Kompleksy $V_\ast, W_\ast$ są 
		\emph{$\ell_2(G)$-homotopijne}, jeśli istnieją morfizmy
		${\phi_\ast:V_\ast \to W_\ast}, {\psi_\ast:V_\ast \to W_\ast}$
		takie, że $\phi_\ast \psi_\ast$ i~$\psi_\ast \phi_\ast$
		są $\ell_2(G)$-homotopijne z~identycznością.
	\end{enumerate}

\end{definicja}

\begin{wniosek}
	Morfizm $\phi_\ast:V_\ast \to W_\ast$ indukuje ograniczone
	$G$-ekwiwariantne operatory 
	$\bar{H}_i(V_\ast) \to \bar{H}_i(W_\ast)$,
	które zależą jedynie od klasy homotopii $\phi_\ast$.
\end{wniosek}

\begin{wniosek}
	Jeśli $\ell_2(G)$-kompleksy łańcuchowe $V_\ast, W_\ast$ są 
	$\ell_2(G)$-homotopijne, to moduły Hilbertowskie
	$\bar{H}_i(V_\ast), \bar{H}_i(W_\ast)$
	są izomorficzne dla wszystkich $i$.
\end{wniosek}

\begin{definicja}
	Ciąg $U \to V \to W$ $G$-modułów Hilbertowskich nazywa się
	krótkim ciągiem słabo-dokładnym, jeśli
	$0\to U \to V \to W \to 0$ jest słabo-dokładnym
	$\ell_2(G)$-kompleksem łańcuchowym.
\end{definicja}

\begin{fakt}
	Jeśli $\alpha:V \to W$ to $G$-ekwiwariantny operator 
	$G$-modułów Hilbertowskich, to $G$-moduły Hilbertowskie
	$\overline{\alpha(V)} \subseteq W$, 
	$(\ker \alpha)^\perp \subseteq V$ są izomorficzne
	oraz $(\ker \alpha)^\perp \simeq \bigslant V {\ker \alpha}$.
\end{fakt}

\begin{wniosek}
	$\dim_G V = {\dim_G \ker \alpha + \dim_G \overline{\alpha(V)}}
	= {\dim_G \ker \alpha + \dim_G \left( \bigslant V {\ker \alpha} \right) }$
\end{wniosek}

\begin{wniosek}
	$U \to V \to W$ słabo-dokładny krótki ciąg 
	$G$-modułów Hilbertowskich, to 
	$\dim_G V = \dim_G U + \dim_G W$.
\end{wniosek}

\begin{wniosek}
	Jeśli $V_\ast : 0 \to V_n \to  \ldots \to V_0 \to 0$ jest
	$\ell_2(G)$-kompleksem łańcuchowym $G$-modułów Hilbertowskich,
	to ${\sum_i (-1)^i \dim_G V_i = \sum_i (-1)^i \dim_G \bar{H}_i(V_\ast)}$.
\end{wniosek}

$Y$ to $\Delta$-kompleks z~wolnym kozwartym symplicjalnym 
działaniem grupy $G$.

\begin{stwierdzenie}
	$C_\ast(Y) = \ell_2(G) \otimes_G K_\ast(Y)$ jest 
	$\ell_2(G)$-kompleksem łańcuchowym.
\end{stwierdzenie}

\begin{definicja}[liczby Bettiego]
	$\beta_i(Y, G) = \dim_G \bar{H}_i(Y)$.
\end{definicja}

\begin{stwierdzenie}
	\begin{enumerate}
		\item $\beta_i(Y, G)$ jest niezmiennikiem $G$-homotopii $Y$,
		jest też niezmiennikiem homotopii $X = \bigslant Y G$.
		
		\item Jeśli $S \subset G$ podgrupa indeksu $m$,
		to $\beta_i(Y, S) = m \beta_i(Y, G)$.
		
		\item $|G| < \infty$, to $\beta_i(Y, G) 
		= \frac 1 {|G|} b_i(Y)$.
		
		W szczególności dla $Y$ spójnego $\beta_0(Y, G)
		= \frac{1}{|G|}$.
		
		\item Jeśli $|G| = \infty$, $Y$ spójny, to $\beta_0(Y, G) = 0$.
	\end{enumerate}
\end{stwierdzenie}

\begin{definicja}[liczby Bettiego]
	$X$ spójny skończony $\Delta$-kompleks,
	$\ell_2$-liczbą Bettiego $\beta_i(X)$ nazywamy
	$\beta_i(\tilde{X}, G)$, gdzie
	$\tilde{X}$ nakrycie uniwersalne,
	$G = \pi_1(X)$ grupa podstawowa $X$.
\end{definicja}







\end{document}
