\documentclass[12pt]{article}
\usepackage{polski}
\usepackage[utf8]{inputenc}
\usepackage[T1]{fontenc}
\usepackage{amsmath}
\usepackage{amsfonts}
\usepackage{fancyhdr}
\usepackage{lastpage}
\usepackage{multirow}
\usepackage{amssymb}
\usepackage{amsthm}
\usepackage{textcomp}
\usepackage{mathabx}
\usepackage{mathtools}
\usepackage{bbm}
\frenchspacing
\usepackage{fullpage}
\setlength{\headsep}{30pt}
\setlength{\headheight}{12pt}
%\setlength{\voffset}{-30pt}
%\setlength{\textheight}{730pt}
\pagestyle{myheadings}

\usepackage{tikz}
%\usepackage{tikz-cd}
\usetikzlibrary{arrows}

\newcommand{\bigslant}[2]{{\left.\raisebox{.2em}{$#1$}\middle/\raisebox{-.2em}{$#2$}\right.}}
\newcommand{\im}{{\mathrm{im}\,}}

\newcommand{\mf}[1]{{\mathfrak{#1}}}
\newcommand{\mb}[1]{{\mathbb{#1}}}
\newcommand{\mc}[1]{{\mathcal{#1}}}
\newcommand{\mr}[1]{{\mathrm{#1}}}

\newcommand{\R}{{\mathbb{R}}}
\newcommand{\Z}{{\mathbb{Z}}}
\newcommand{\N}{{\mathcal{N}}}
\newcommand{\tr}{{\mathrm{tr}}}
\renewcommand{\H}{{\mathcal{H}}}
\newcommand{\1}{{\mathbbm{1}}}
\newcommand{\Hom}{{\mathrm{Hom}}}
\newcommand{\Id}{{\mathrm{Id}}}
\newcommand{\simp}[1]{{\Delta^{#1}}}
\newcommand{\osimp}[1]{{\mathring{\Delta}^{#1}}}

\newcounter{punkt}

\theoremstyle{plain}
\newtheorem{twierdzenie}[punkt]{Twierdzenie}
\newtheorem{twierdzeniebd}[punkt]{Twierdzenie (bez dowodu)}
\newtheorem{lemat}[punkt]{Lemat}
\newtheorem{stwierdzenie}[punkt]{Stwierdzenie}
\newtheorem{hipoteza}[punkt]{Hipoteza}

\theoremstyle{definition}
\newtheorem{definicja}[punkt]{Definicja}
\newtheorem{fakt}[punkt]{Fakt}
\newtheorem{wniosek}[punkt]{Wniosek}

\theoremstyle{remark}
\newtheorem{uwaga}[punkt]{Uwaga}
\newtheorem{przyklad}[punkt]{Przykład}
\newtheorem{cwiczenie}[punkt]{Ćwiczenie}



\markright{Piotr Suwara\hfill $\ell_2$-metody w~topologii: 7 października 2013\hfill}
 
\begin{document}

\begin{definicja}[$n$-sympleks]
	Najmniejszy wypukły zbiór w~$\R^m$ zawierający $n+1$ punktów nie leżących 
	w~hiperprzestrzeni wymiaru $n-1$.
	
	Sympleks o~wierzchołkach $v_i$ oznaczamy $[v_0, \ldots, v_n]$.
\end{definicja}

\begin{definicja}[sympleks standardowy] 
	$\Delta^n = [e_0, \ldots, e_n] $\\
	$= \{ (t_0, \ldots, t_n) \in \R^{n+1} | \sum t_i = 1, t_i \geq 0 \}$.
\end{definicja}

\begin{uwaga}
	Wierzchołki sympleksu są uporządkowane. 
	Ustalenie porządku wyznacza kanoniczny liniowy homeomorfizm
	sympleksu standardowego $\Delta^n$ z~dowolnym sympleksem
	$[v_0, \ldots, v_n]$,
	tj. $\Delta^n \owns (t_0, \ldots, t_n) \mapsto \sum t_i v_i$.
\end{uwaga}

\begin{uwaga}
	Ściany dziedziczą porządek wierzchołków po sympleksie.
\end{uwaga}

\begin{definicja}[brzeg sympleksu]
	Suma wyszystkich ścian sympleksu jest jego brzegiem $\partial \Delta^n$.
\end{definicja}

\begin{definicja}[otwarty sympleks]
	$\osimp n = \simp n \setminus \partial \simp n$
\end{definicja}

\begin{definicja}[$\Delta$-kompleks]
	Przestrzeń topologiczna $X$ ma strukturę $\Delta$-kompleksu, 
	jeśli istnieje rodzina przekształceń $\sigma_\alpha: \Delta^{n_\alpha} \to X$
	takich, że
	\begin{itemize}
 		\item $\sigma_\alpha |_\osimp{n_\alpha}$ jest włożeniem
 		oraz każdy puinkt jest w~obrazie dokładnie jednego
 		z~przekształceń $\sigma_\alpha |_\osimp{n_\alpha}$,
 		\item dowolne obcięcie $\sigma_\alpha$ do ściany sympleksu 
 		jest jednym z~przekształceń $\sigma_\beta$,
 		\item zbiór $A \subseteq X$ jest otwarty wtw, gdy
 		$\sigma_\alpha^{-1}(A)$ jest otwarty w~$\simp n$ dla każdego
 		$\alpha$.
	\end{itemize}

\end{definicja}

\begin{przyklad}
	W~notatkach narysowany podział torusa na komórki.
\end{przyklad}

\begin{definicja}[$\Delta$-kompleks]
	$\Delta$-kompleks $X$ jest przestrzenią ilorazową przestrzeni
	$\bigsqcup \Delta_\alpha^n$, gdzie każdą ścianę $\Delta_\alpha^n$
	identyfikujemy z~odpowiednim $\Delta_\beta^{n-1}$
	odpowiadającym obcięciu ${\sigma_\beta = \sigma_\alpha |_{sciana}}$.
\end{definicja}

\begin{uwaga}
	Możemy też konstruować $X$ indukcyjnie:
	$X^{(0)}$ to dyskretny zbiór wierzchołków,
	$X^{(1)}$ z~doklejonymi krawędziami itd.
\end{uwaga}
\pagebreak
{\bf Homologie simplicjalne}
$X$ to $\Delta$-kompleks, $G$ to grupa abelowa (domyślnie $\Z$).

\begin{definicja}[$n$-łańcuchy]
	$\Delta_n(X,G)$ to wolna grupa abelowa, 
	której bazą są otwarte $n$-sympleksy $e_\alpha^n$ w~$X$:
	$\Delta_n(X,G) 
	= \left\{ \sum_{\alpha sk.} n_\alpha e_\alpha^n : n_\alpha \in G \right\}$.
	
	Jej elementy nazywamy $n$-łańcuchami.
\end{definicja}

\begin{uwaga}
	$[v_0, \ldots, v_i, v_{i+1}, \ldots, v_n]
	= - [v_0, \ldots, v_{i+1}, v_i, \ldots, v_n]$
\end{uwaga}

\begin{definicja}[brzeg]
	$\partial [v_0, \ldots, v_n] 
	= \sum (-1)^i [v_0, \ldots, \hat{v}_i, \ldots, v_n]$
	
	$\partial_n(\sigma_\alpha)
	= \sum (-1)^i \sigma_\alpha |_{[v_0, \ldots, \hat{v}_i, \ldots, v_n]}$
\end{definicja}

\begin{lemat}
	$\partial^2 = 0$
\end{lemat}

\begin{definicja}
	Mamy {\em kompleks łańcuchowy} $\Delta(X,G)$, \\
	$\ldots \to \Delta_n(X, G) \xrightarrow{\partial_n}
	\Delta_{n-1}(X,G) \to \ldots \to \Delta_0(X,G) \to 0$.
\end{definicja}

\begin{definicja}[homologie simplicjalne]
	$\displaystyle H_n^\Delta(X,G) = H_n(\Delta(X,G)) 
	= \bigslant{\ker \partial_n}{\im \partial_{n+1}}$
	
	Elementy $\ker \partial_n$ nazywamy \emph{cyklami},
	$\im \partial_{n+1}$ -- \emph{brzegami},
	$H_n(X,G)$ -- \emph{klasami homologii}.
\end{definicja}

{\bf Homologie singularne}

\begin{definicja}
	$C_n(X,G)$ -- wolna grupa abelowa generowana przez 
	zbiór singularnych $n$-sympleksów w~$X$ 
	-- kompleks łańcuchowy singularny.
	
	$\partial_n(\sigma) 
	= \sum(-1)^i \sigma|_{[v_0, \ldots, \hat{v}_i, \ldots, v_n]}$
	
	$\partial^2 = 0$
	
	$H_n(X,G) = \bigslant{\ker \partial_n}{\im \partial_{n+1}}$
\end{definicja}

\begin{uwaga}
	Homologie singularne są szczególnym przypadkiem simplicjalnych
	(wyjaśnienie w~notatkach).
\end{uwaga}

\begin{stwierdzenie}
	$H_n(X) = \bigoplus_\alpha H_n(X_\alpha)$, 
	gdzie $X_\alpha$ -- składowe łukowej spójności przestrzeni $X$.
\end{stwierdzenie}

\begin{stwierdzenie}
	$H_0(X,G) = \bigoplus_\alpha G$, 
	$\alpha$ indeksują składowe łukowej spójności przestrzeni $X$.
\end{stwierdzenie}

\begin{stwierdzenie}
	$H_n(\ast, G) = \begin{cases} G & n = 0 \\ 0 & n \geq 1 \end{cases}$
\end{stwierdzenie}

\begin{definicja}[przekształcenie indukowane]
	$f:X \to Y$ ciągłe indukuje ${f_\# : C_n(X,G) \to C_n(Y,G)}$
	takie, że $\sigma \mapsto f \circ \sigma$.
\end{definicja}

\begin{lemat}
	$F_\# \partial = \partial f_\#$
\end{lemat}

\begin{wniosek}
	$f_\#$ przenosi brzegi na brzegi, cykle na cykle, czyli
	$f$ indukuje ${f_\ast : H_n(X,G) \to H_n(Y,G)}$ dla każdego $n$.
\end{wniosek}

\begin{fakt}
	$(fg)_\ast = f_\ast g_\ast$
\end{fakt}

\begin{fakt}
	$\left(\mr{Id}_X\right)_\# = \mr{Id}_{H_n}$
\end{fakt}

\begin{twierdzenie}
	$f \simeq_{htp} g : X \to Y \implies f_\ast = g_\ast$
\end{twierdzenie}

\begin{wniosek}
	$X \simeq_{htp} Y \implies H_\ast(X, G) = H_\ast(Y,G)$
\end{wniosek}

\begin{twierdzenie}
	$H_n^{sing} \simeq H_n^\Delta$
\end{twierdzenie}

\begin{definicja}[kompleksy]
	$\ldots \to C_{n+1} \xrightarrow{\partial} C_n \to \ldots$ to kompleks łańcuchowy, \\
	$\ldots \to C^{n+1} \xleftarrow{\delta} C^n \to \ldots$ to kompleks kołańcuchowy.
	
	Mając kompleks łańcuchowy $(C_n)$ możemy wziąć 
	$C^n = \Hom(C_n, G) = C_n^\ast, \delta = \partial^\ast$.
	
	Wtedy też $\delta^2 = 0$.
\end{definicja}

\begin{definicja}[kompleks kołańcuchowy singularny]
	$C^n(X,H) = \Hom(C_n(X,G), G)$
	
	$\displaystyle (\delta \varphi)(\sigma) 
	= \sum_{i=0}^n (-1)^i \varphi
	\left(\sigma|_{[v_0, \ldots, \hat{v}_i, \ldots, v_n]} \right)$
\end{definicja}

\begin{definicja}[kohomologie singularne]
	$H^n(X,G) = \bigslant{\ker \delta}{\im \delta}$
	
	$\ker \delta$ to kocykle, $\im \delta$ to kobrzegi.
\end{definicja}

\begin{uwaga}
	$\varphi \in C^n(X,G)$ kocykl, jeśli 
	$0 = \delta \varphi = \varphi \delta$, 
	czyli jeśli $\varphi$ znika na brzegach.
\end{uwaga}

\begin{uwaga}
	Jeśli $X$ -- skończony $\Delta$-kompleks, 
	to utożsamiamy $\Delta_n(X, \R)$ 
	i~$\Hom(\Delta_n(X,\R), \R) = \Delta_n(X, \R)^\ast$.
\end{uwaga}

\begin{definicja}
	$\Delta_n(X) = \Delta_n(X,\R)$, $\Delta^n(X) = \Delta^n(X,\R)$ etc.
\end{definicja}


{\bf Rozkład Hodge'a - de Rhama}

$X$ skończony $\Delta$-kompleks,
$\Delta^n(X, \R) \cong \R^{\beta_i}$ przestrzenie liniowe
skończonego wymiaru nad $\R$ ($\beta_i$ -- liczba $n$-sympleksów)
mają naturalny iloczyn skalarny

\begin{definicja}
	$\langle f, f' \rangle = \sum f(e_\alpha^n) f'(e_\alpha^n)$
\end{definicja}

\begin{stwierdzenie}
	$\langle \delta_{i-1} x, y \rangle = \langle x, \partial_i y \rangle$
	(de facto -- z~definicji)
\end{stwierdzenie}

\begin{wniosek}[równoważna definicja]
	$\delta_{n-1} = \partial_n^\ast$
	
	$\Delta^n(X) \xleftrightharpoons[\partial_n]{\delta_{n-1} = \partial_n^\ast} \Delta^{n-1}(X)$
\end{wniosek}

\begin{wniosek}
	$Z^i = \ker \delta_i = (\im \partial_{i+1})^\perp = B_i^\perp$
	
	$Z_i = \ker \partial_i = (\im \delta_{i-1})^\perp = {B^i}^\perp$
	
	$\Delta^i(X) = B^i \operp Z_i = B_i \operp Z^i$
\end{wniosek}

\begin{stwierdzenie}
	$B^i \perp B_i$
\end{stwierdzenie}

\begin{twierdzenie}[rozkład H-dR]
	$\Delta_i(X) = B^i \operp B_i \operp (Z^i \cap Z_i)$
\end{twierdzenie}

\begin{definicja}[harmoniczne kołańcuchy]
	$\H_i(X) = Z_i(X) \cap Z^i(X)$
\end{definicja}

\begin{definicja}[laplasjan]
	$\Delta_i = \partial_{i+1} \delta_i + \delta_{i-1} \partial_i 
	: \Delta^i(X) \to \Delta^i(X)$
\end{definicja}

\begin{lemat}
	$\H_i(X) = \ker \Delta_i$
\end{lemat}



\end{document}
 
 
 
 
 
 
