\documentclass[12pt]{article}
\usepackage{polski}
\usepackage[utf8]{inputenc}
\usepackage[T1]{fontenc}
\usepackage{amsmath}
\usepackage{amsfonts}
\usepackage{fancyhdr}
\usepackage{lastpage}
\usepackage{multirow}
\usepackage{amssymb}
\usepackage{amsthm}
\usepackage{textcomp}
\usepackage{mathabx}
\usepackage{mathtools}
\usepackage{bbm}
\frenchspacing
\usepackage{fullpage}
\setlength{\headsep}{30pt}
\setlength{\headheight}{12pt}
%\setlength{\voffset}{-30pt}
%\setlength{\textheight}{730pt}
\pagestyle{myheadings}

\usepackage{tikz}
%\usepackage{tikz-cd}
\usetikzlibrary{arrows}

\newcommand{\bigslant}[2]{{\left.\raisebox{.2em}{$#1$}\middle/\raisebox{-.2em}{$#2$}\right.}}
\newcommand{\im}{{\mathrm{im}\,}}

\newcommand{\mf}[1]{{\mathfrak{#1}}}
\newcommand{\mb}[1]{{\mathbb{#1}}}
\newcommand{\mc}[1]{{\mathcal{#1}}}
\newcommand{\mr}[1]{{\mathrm{#1}}}

\newcommand{\R}{{\mathbb{R}}}
\newcommand{\Z}{{\mathbb{Z}}}
\newcommand{\N}{{\mathcal{N}}}
\newcommand{\tr}{{\mathrm{tr}}}
\renewcommand{\H}{{\mathcal{H}}}
\newcommand{\1}{{\mathbbm{1}}}
\newcommand{\Hom}{{\mathrm{Hom}}}
\newcommand{\Id}{{\mathrm{Id}}}
\newcommand{\simp}[1]{{\Delta^{#1}}}
\newcommand{\osimp}[1]{{\mathring{\Delta}^{#1}}}

\newcounter{punkt}

\theoremstyle{plain}
\newtheorem{twierdzenie}[punkt]{Twierdzenie}
\newtheorem{twierdzeniebd}[punkt]{Twierdzenie (bez dowodu)}
\newtheorem{lemat}[punkt]{Lemat}
\newtheorem{stwierdzenie}[punkt]{Stwierdzenie}
\newtheorem{hipoteza}[punkt]{Hipoteza}

\theoremstyle{definition}
\newtheorem{definicja}[punkt]{Definicja}
\newtheorem{fakt}[punkt]{Fakt}
\newtheorem{wniosek}[punkt]{Wniosek}

\theoremstyle{remark}
\newtheorem{uwaga}[punkt]{Uwaga}
\newtheorem{przyklad}[punkt]{Przykład}
\newtheorem{cwiczenie}[punkt]{Ćwiczenie}



\markright{Piotr Suwara\hfill $\ell_2$-metody w~topologii: 27 stycznia 2014 \hfill}
 
\begin{document}

\begin{stwierdzenie}
	$\| d \omega\|_{k-1}^2 \leq \| \omega \_k^2$
\end{stwierdzenie}

\begin{wniosek}
	$d : A^{k,p} \to A^{k-1, p+1}$ jest ograniczonym operatorem
	dla każdego $k \geq 1, p \geq 0$.
\end{wniosek}

\begin{definicja}
	Rozważmy kompleks
	${0 \to A^{t,0} \xrightarrow d A^{t-1,1} \xrightarrow d \ldots
	\xrightarrow d A^{t-N, N} \to 0}$ (${N = \dim X = \dim \tilde{X}}$)
	i~kohomologie:
	
	$Z^{k,p} = \{ \omega \in A^{k,p} | d\omega = 0\}$
	
	$B^{k,p} = \im d : A^{k+1, p-1} \to A^{k,p}$
	
	$H^{k,p} = \bigslant{Z^{k,p}}{\overline{B^{k,p}}^{A^{k,p}}}$
\end{definicja}

\begin{stwierdzenie}
	Przestrzenie $H^{k,p}$ są niezależne od wyboru $k$:
	$\forall_{k \geq 0, p \geq 0} H^{k,p}$ jest $\Gamma$-izomorficzne
	z~$\H^p(\tilde{X})$.
\end{stwierdzenie}

\begin{definicja}
	$\| \omega \|_m^U = \left( \int_U \left|
		(\Id + \Delta)^m \omega ( x) \right|^2 dV \right)^{\frac12}$
	dla każdego zbioru otwartego ${U \subset \tilde{X}}$, 
	${\omega \in A^{k,p}}$, ${1 \leq m \leq k}$.
\end{definicja}

\begin{lemat}
	Niech $(V, x^1, \ldots, x^N)$ będzie układem współrzędnych
	w~$\tilde{X}$, $U$ otwarty, relatywnie zwarty zbiór,
	$\bar{U} \subseteq V$.
	Dla $k \geq \frac N 4 + \frac 1 2$ oraz każdego $\omega \in A^{k,p}$
	forma $\omega$ jest klasy $C^1$. Ponadto istnieje $C>0$
	niezależne od $\omega$ takie, że
	$$\sup_{x \in U} |\omega(x)| \leq C( \| \omega \|_k^V + 
		\|\omega\|_0^V),$$
	$$\max_{1 \leq i \leq N} \sup_{x \in U}
		\left| \frac{ \partial \omega }{\partial x^i} (x) \right|
		\leq C (\| \omega \|_k^V + \| \omega \|_0^V ),$$
	gdzie $\frac{\partial \omega}{\partial x_i}$ to $\omega$
	ze zróżniczkowanymi współczynnikami.
\end{lemat}

Niech $\tilde{K}$ -- kompleks taki, że $| \tilde{K}| = \tilde{X}$
(gładka triangulacja).

\begin{definicja}
	$\omega \in \Omega^p(\tilde{X})$, definiujemy
	${\int \omega \in C^p(\tilde{K})}$, 
	${\int \omega = \sum (\int_\sigma \omega) \sigma}$ -- suma
	po $p$-sympleksach w~$\tilde{K}$.
\end{definicja}

\begin{lemat}
	Niech $\omega \in A^{k,p}$, $k > \frac N 4 + \frac 1 2$.
	
	Wówczas $\int \omega$ jest $L_2$.
	
	Co więcej, $\int : A^{k,p} \to C_{(2)}^p(\tilde{K})$ 
	jest ograniczony
	\\ oraz $\int B^{k,p} \subseteq d_C C_{(2)}^{p-1} (\tilde{K})$.
\end{lemat}

\begin{wniosek}
	$\int$ indukuje przekształcenie 
	${H^p(\tilde{X}) \to H_{(2)}^p(\tilde{K})}$.
\end{wniosek}
\pagebreak
{\bf Konstrukcja Whitneya}

$\{ U_v \}_{v \in K^0}$ otwarte pokrycie $X$ otwartymi gwiazdami
wierzchołków: $U_v = \mr{star}\,v$.

Weźmy gładki rozkład jedności stowarzyszony z~tym pokryciem.
Zarówno pokrycie, jak i~rozkład jedynki możemy podnieść do $\tilde{X}$:
${\{ \varphi_v \}_{v \in \tilde{K}^0}}$, 
${\mr{supp}\,\varphi_v \subset \mr{star}\, v}$,
${\varphi_v \gamma = \varphi_{\gamma^{-1} v}}$.

Niech $\sigma = [v_0, \ldots, v_p]$ sympleks w~$\tilde{K}$,
oznaczamy $\varphi_i = \varphi_{v_i}$.

\begin{definicja}
	$W_\sigma = \begin{cases}
	            	\varphi_0 & \text{dla }p=0 \\
	            	p! \left( \sum_{i=0}^p (-1)^i \varphi_i
	            	d \varphi_0 \wedge \ldots \wedge 
	            	\widehat{ d \varphi_i} \wedge \ldots
	            	\wedge d \varphi_p \right)
	            	& \text{dla }p \geq 1
	            \end{cases}$
\end{definicja}

\begin{uwaga}
	$\gamma^\ast W_\sigma = W_{\gamma^{-1} \sigma}$
\end{uwaga}

\begin{uwaga}
	$\mr{supp}\,W_\sigma \subseteq \mr{star}\, \sigma$
\end{uwaga}

\begin{definicja}
	Dla $f = \sum_{\sigma \in \tilde{K}} f_\sigma \sigma$
	bierzemy $$W_f = \sum_{\sigma \in \tilde{K}} f_\sigma W_\sigma.$$
\end{definicja}

\begin{stwierdzenie}
	\begin{itemize}
		\item $d W = W d_C$
		\item $\int \circ W = \Id$
		\item $\forall_{f \in C^\ast(\tilde{K}), \gamma \in \Gamma} 
		\gamma^\ast W f = W(f \gamma)$
	\end{itemize}

\end{stwierdzenie}

\begin{lemat}
	$\forall_{f \in C_{(2)}^p(\tilde{K}), k \geq 0}$
	gładka forma $Wf \in A^{k,p}$.
	
	Co więcej, $\forall_{k \geq 0} W:C_{(2)}^p(\tilde{K}) \to A^{k,p}$
	jest ograniczony i~$\Gamma$-ekwiwariantny.
\end{lemat}

\begin{wniosek}
	$W$ indukuje przekształcenie 
	$W: H_{(2)}^p(\tilde{K}) \to H^p(\tilde{X})$.
\end{wniosek}


Na koniec było trochę o~hipotezach itp.



\end{document}
