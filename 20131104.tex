\documentclass[12pt]{article}
\usepackage{polski}
\usepackage[utf8]{inputenc}
\usepackage[T1]{fontenc}
\usepackage{amsmath}
\usepackage{amsfonts}
\usepackage{fancyhdr}
\usepackage{lastpage}
\usepackage{multirow}
\usepackage{amssymb}
\usepackage{amsthm}
\usepackage{textcomp}
\usepackage{mathabx}
\usepackage{mathtools}
\usepackage{bbm}
\frenchspacing
\usepackage{fullpage}
\setlength{\headsep}{30pt}
\setlength{\headheight}{12pt}
%\setlength{\voffset}{-30pt}
%\setlength{\textheight}{730pt}
\pagestyle{myheadings}

\usepackage{tikz}
%\usepackage{tikz-cd}
\usetikzlibrary{arrows}

\newcommand{\bigslant}[2]{{\left.\raisebox{.2em}{$#1$}\middle/\raisebox{-.2em}{$#2$}\right.}}
\newcommand{\im}{{\mathrm{im}\,}}

\newcommand{\mf}[1]{{\mathfrak{#1}}}
\newcommand{\mb}[1]{{\mathbb{#1}}}
\newcommand{\mc}[1]{{\mathcal{#1}}}
\newcommand{\mr}[1]{{\mathrm{#1}}}

\newcommand{\R}{{\mathbb{R}}}
\newcommand{\Z}{{\mathbb{Z}}}
\newcommand{\N}{{\mathcal{N}}}
\newcommand{\tr}{{\mathrm{tr}}}
\renewcommand{\H}{{\mathcal{H}}}
\newcommand{\1}{{\mathbbm{1}}}
\newcommand{\Hom}{{\mathrm{Hom}}}
\newcommand{\Id}{{\mathrm{Id}}}
\newcommand{\simp}[1]{{\Delta^{#1}}}
\newcommand{\osimp}[1]{{\mathring{\Delta}^{#1}}}

\newcounter{punkt}

\theoremstyle{plain}
\newtheorem{twierdzenie}[punkt]{Twierdzenie}
\newtheorem{twierdzeniebd}[punkt]{Twierdzenie (bez dowodu)}
\newtheorem{lemat}[punkt]{Lemat}
\newtheorem{stwierdzenie}[punkt]{Stwierdzenie}
\newtheorem{hipoteza}[punkt]{Hipoteza}

\theoremstyle{definition}
\newtheorem{definicja}[punkt]{Definicja}
\newtheorem{fakt}[punkt]{Fakt}
\newtheorem{wniosek}[punkt]{Wniosek}

\theoremstyle{remark}
\newtheorem{uwaga}[punkt]{Uwaga}
\newtheorem{przyklad}[punkt]{Przykład}
\newtheorem{cwiczenie}[punkt]{Ćwiczenie}



\markright{Piotr Suwara\hfill $\ell_2$-metody w~topologii: 4 listopada 2013\hfill}
 
\begin{document}

$X$ przestrzeń topologiczna, $A \subseteq X$ podprzestrzeń,
to $C_n(A) \hookrightarrow C_n(X)$.

\begin{definicja}[homologie relatywne]
	$C_n(X, A) = \bigslant{C_n(X)}{C_n(A)}$
\end{definicja}

\begin{definicja}[operator brzegu]
	$\partial:C_n(X) \to C_{n-1}(X)$
	przeprowadza $C_n(A)$ w~$C_{n-1}(A)$,
	czyli indukuje $\partial: C_n(X, A) \to C_{n-1}(X, A)$.
\end{definicja}

\begin{definicja}[relatywne homologie]
	$H_n(X, A) = H_n( C_n(X, A), \partial)$
\end{definicja}

\begin{wniosek}
	Elementy $H_n(X, A)$ są reprezentowane przez $n$-łańcuchy
	$\alpha \in C_n(X)$, dla których $\partial \alpha \in C_{n-1}(A)$.
\end{wniosek}

\begin{wniosek}
	Relatywny cykl $\alpha$ jest trywialny, jeśli jest relatywnym
	brzegiem: $\alpha = \partial \beta + \gamma$
	dla pewnych $\beta \in C_{n+1}(X), \gamma \in C_n(A)$.
\end{wniosek}

\begin{twierdzenie}
	Ciąg $\ldots \to H_n(A) 
	\xrightarrow{i_\ast} H_n(X) 
	\xrightarrow{j_\ast} H_n(X, A)
	\xrightarrow{\tau} H_{n-1}(A) \to \ldots$
	jest dokładny.
	
	Weźmy $[c] \in C_n(X, A)$ dla $c \in C_n(X)$, takie, że
	$\partial [c] = 0$,
	wtedy $\tau([c]) = \partial c \subset C_{n-1}(A)$.
\end{twierdzenie}

\begin{twierdzenie}
	$f:(X, A) \to (Y, B)$ indukuje $f_\ast: H_n(X, A) \to H_n(Y, B)$
	i~$f_\ast$ są przemienne z~$\tau$, czyli
	$\tau$ jest naturalną transformacją z~$H_n(X, A)$ do $H_{n-1}(A)$.
\end{twierdzenie}

\begin{twierdzenie}
	Kanoniczny homeomorfizm $H_n^\Delta(X) \to H_n(X)$ jest
	izomorfizmem dla każdego $n$ i~dla każdego $\Delta$-kompleksu $X$.
\end{twierdzenie}

\begin{cwiczenie}
	Udowodnić równość $\ell_2$-homologii zredukowanych 
	dla różnych struktur $\Delta$-kompleksu.
	
	Ogólniej: homotopijną niezmienniczośc $\ell_2$-homologii
	zredukowanych.
\end{cwiczenie}

\begin{cwiczenie}
	$Y = \R^n, G = \Z^n, X=T^n = (S^1)^n$
	
	Udowodnij, że $\N(\Z^n) \simeq L_\infty(T^n)$
	i~znajdź wzór na $\tr_G: L_\infty(T^n) \to \R$.
	
	Podpowiedź w~notatkach.
\end{cwiczenie}










\end{document}
