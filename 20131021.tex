\documentclass[12pt]{article}
\usepackage{polski}
\usepackage[utf8]{inputenc}
\usepackage[T1]{fontenc}
\usepackage{amsmath}
\usepackage{amsfonts}
\usepackage{fancyhdr}
\usepackage{lastpage}
\usepackage{multirow}
\usepackage{amssymb}
\usepackage{amsthm}
\usepackage{textcomp}
\usepackage{mathabx}
\usepackage{mathtools}
\usepackage{bbm}
\frenchspacing
\usepackage{fullpage}
\setlength{\headsep}{30pt}
\setlength{\headheight}{12pt}
%\setlength{\voffset}{-30pt}
%\setlength{\textheight}{730pt}
\pagestyle{myheadings}

\usepackage{tikz}
%\usepackage{tikz-cd}
\usetikzlibrary{arrows}

\newcommand{\bigslant}[2]{{\left.\raisebox{.2em}{$#1$}\middle/\raisebox{-.2em}{$#2$}\right.}}
\newcommand{\im}{{\mathrm{im}\,}}

\newcommand{\mf}[1]{{\mathfrak{#1}}}
\newcommand{\mb}[1]{{\mathbb{#1}}}
\newcommand{\mc}[1]{{\mathcal{#1}}}
\newcommand{\mr}[1]{{\mathrm{#1}}}

\newcommand{\R}{{\mathbb{R}}}
\newcommand{\Z}{{\mathbb{Z}}}
\newcommand{\N}{{\mathcal{N}}}
\newcommand{\tr}{{\mathrm{tr}}}
\renewcommand{\H}{{\mathcal{H}}}
\newcommand{\1}{{\mathbbm{1}}}
\newcommand{\Hom}{{\mathrm{Hom}}}
\newcommand{\Id}{{\mathrm{Id}}}
\newcommand{\simp}[1]{{\Delta^{#1}}}
\newcommand{\osimp}[1]{{\mathring{\Delta}^{#1}}}

\newcounter{punkt}

\theoremstyle{plain}
\newtheorem{twierdzenie}[punkt]{Twierdzenie}
\newtheorem{twierdzeniebd}[punkt]{Twierdzenie (bez dowodu)}
\newtheorem{lemat}[punkt]{Lemat}
\newtheorem{stwierdzenie}[punkt]{Stwierdzenie}
\newtheorem{hipoteza}[punkt]{Hipoteza}

\theoremstyle{definition}
\newtheorem{definicja}[punkt]{Definicja}
\newtheorem{fakt}[punkt]{Fakt}
\newtheorem{wniosek}[punkt]{Wniosek}

\theoremstyle{remark}
\newtheorem{uwaga}[punkt]{Uwaga}
\newtheorem{przyklad}[punkt]{Przykład}
\newtheorem{cwiczenie}[punkt]{Ćwiczenie}



\markright{Piotr Suwara\hfill $\ell_2$-metody w~topologii: 21 października 2013\hfill}
 
\begin{document}

\begin{uwaga}[gwoli uzupełnienia]
	$H^i_G(Y, \ell_2(G)) = \bigslant{Z^i(Y)}{B^i(Y)}$
	
	$\bar{H}^n(Y) = \bigslant{Z^i(Y)}{\bar{B}^i(Y)} \simeq 
	\H_n(Y) = \ker \Delta_n$
\end{uwaga}

{\bf Moduły Hilbertowskie}

\begin{definicja}[$G$-moduł Hilbertowski]
	$M$ przestrzeń Hilberta nazywa się \emph{$G$-modułem Hilbertowskim},
	jeśli $M$ jest wyposażona w~reprezentację unitarną grupy $G$,
	$\pi:G \to U(M)$, taką, że $M$ jest izometrycznie $G$-izomorficzna
	z~domkniętą $G$-niezmienniczą podprzestrzenią $(\ell_2(G))^n$
	(istnieje $V \subseteq (\ell_2(G))^n$ domknięta, $T:M \to V$ 
	liniowa izometria, $T \pi_g = \lambda_g T$.
\end{definicja}

\begin{lemat}
	$\delta_i \lambda_g = \lambda_g \delta_i$
\end{lemat}


\begin{fakt}
	$C_i(Y) \simeq (\ell_2(G))^{\alpha_i}$ jest $G$-modułem Hilbertowskim.
	
	$Z^i(Y), Z_i(Y)$ są $G$-modułami Hilbertowskimi.
	
	$\bar{B}_i(Y), \bar{B}^i(Y)$ są $G$-modułami Hilbertowskimi.
\end{fakt}

\begin{wniosek}
	$\H_i(Y)$ jest $G$-modułem Hilbertowskim.
\end{wniosek}

\begin{lemat}
	$M$ -- $G$-moduł Hilbertowski, 
	$V \subseteq M$ to $G$-niezmiennicza podprzestrzeń $M$,
	wówczas $\bigslant M {\bar{V}}$
	z~normą $\| w \| = \inf \{ \| \tilde{w} \| : \pi(\tilde{w}) = w \}$
	ma naturalną strukturę $G$-modułu Hilbertowskiego.
\end{lemat}

\begin{lemat}
	Jeśli $\bar{V} \subseteq M$ jest $G$-niezmiennicza,
	to $\bar{V}^\perp$ jest $G$-niezmiennicze.
\end{lemat}

\begin{definicja}[izomorfizmy]
	$f:M_1 \to M_2$ przekształcenie $G$-modułów Hilbertowskich jest
	\begin{itemize}
		\item \emph{słabym izomorfizmem}, jeśli jest injekcją,
		ograniczone, $G$-ekwiwariantne, 
		oraz $\im f$ jest gęste w~$M_2$;
		\item \emph{silnym izomorfizmem}, jeśli jest
		$G$-ekwiwariantną izometrią $M_1$ i~$M_2$.
	\end{itemize}
\end{definicja}

\begin{lemat}
	Załóżmy, że istnieje słaby izomorfizm $G$-modułów Hilbertowskich
	$M_1 \to M_2$. Wtedy istnieje silny izomorfizm $M_1$ i~$M_2$.
\end{lemat}

\begin{uwaga}
	$T:H_1 \to H_2$, $\im T$ gęsty w~$H_2$ wtw, 
	gdy $T^\ast$ jest injekcją, bo $(\im T)^\perp = \ker T^\ast$.
\end{uwaga}

\begin{twierdzenie}[spektralne]
	$T \in \mc{B}(H)$ dodatni samosprzężony, 
	to istnieje $S \in \mc{B}(H)$ dodatni samosprzężony taki, że
	$S^2 = T$, co więcej $S$ jest przemienny z~każdym operatorem, 
	z~którym przemienny jest $T$.
\end{twierdzenie}

\begin{definicja}[izomorficzne $G$-moduły]
	$M_1, M_2$ -- $G$-moduły Hilbertowskie są \emph{izomorficzne},
	jeśli istnieje słaby (lub równoważnie -- silny) izomorfizm
	$f : M_1 \to M_2$.
\end{definicja}

\begin{wniosek}
	$\varphi: M_1 \to M_2$ ograniczone $G$-ekwiwariantne
	przekształcenie $G$-modułów Hilbertowskich,
	wówczas $(\ker \varphi) ^\perp 
	\simeq \bigslant{M_1}{\ker \varphi} \simeq \overline{\im \varphi}$
	jako $G$-moduły Hilbertowskie.
\end{wniosek}
\pagebreak 
\begin{twierdzenie}
	$\H_i$ jest funktorem z~kategorii $\Delta$-kompleksów
	z~wolnym kozwartym działaniem $G$ i~klas $G$-homotopii
	przekształceń do kategorii $G$-modułów Hilbertowskich
	i~ograniczonych $G$-ekwiwariantnych operatorów.
\end{twierdzenie}

\begin{twierdzenie}[o~aproksymacji symplicjalnej]
	Niech $K$ -- skończony $\Delta$-kompleks, 
	$L$ -- dowolny $\Delta$-kompleks.
	Wówczas przekształcenie $f:K \to L$ jest homotopijne
	z~przekształceniem symplicjalnym $f':K' \to L$,
	gdzie $K'$ jest pewnym podpodziałem barycentrycznym $K$.
\end{twierdzenie}

\begin{definicja}
	Przekształcenie symplicjalne przekształca sympleksy na sympleksy 
	i~jest liniowe w~obcięciu do wnętrza każdego sympleksu.
\end{definicja}

\begin{lemat}
	W~$(\ell_2(G))^n$ nie ma $G$-niezmienniczych niezerowych wektorów, jeśli $|G| = \infty$.
\end{lemat}

\begin{przyklad}
	$Y$ -- spójny $G$-$\Delta$-kompleks z~kozwartym $1$-szkieletem,
	$G$ nieskończona, wtedy $\H_0(Y) = 0$.
\end{przyklad}

\begin{przyklad}
	Dla $Y = \R, G = \Z, X = S^1$ 
	mamy $\H_0(Y) = 0 \neq H_0^G(Y, \ell_2(G))$.
\end{przyklad}













\end{document}
