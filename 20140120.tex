\documentclass[12pt]{article}
\usepackage{polski}
\usepackage[utf8]{inputenc}
\usepackage[T1]{fontenc}
\usepackage{amsmath}
\usepackage{amsfonts}
\usepackage{fancyhdr}
\usepackage{lastpage}
\usepackage{multirow}
\usepackage{amssymb}
\usepackage{amsthm}
\usepackage{textcomp}
\usepackage{mathabx}
\usepackage{mathtools}
\usepackage{bbm}
\frenchspacing
\usepackage{fullpage}
\setlength{\headsep}{30pt}
\setlength{\headheight}{12pt}
%\setlength{\voffset}{-30pt}
%\setlength{\textheight}{730pt}
\pagestyle{myheadings}

\usepackage{tikz}
%\usepackage{tikz-cd}
\usetikzlibrary{arrows}

\newcommand{\bigslant}[2]{{\left.\raisebox{.2em}{$#1$}\middle/\raisebox{-.2em}{$#2$}\right.}}
\newcommand{\im}{{\mathrm{im}\,}}

\newcommand{\mf}[1]{{\mathfrak{#1}}}
\newcommand{\mb}[1]{{\mathbb{#1}}}
\newcommand{\mc}[1]{{\mathcal{#1}}}
\newcommand{\mr}[1]{{\mathrm{#1}}}

\newcommand{\R}{{\mathbb{R}}}
\newcommand{\Z}{{\mathbb{Z}}}
\newcommand{\N}{{\mathcal{N}}}
\newcommand{\tr}{{\mathrm{tr}}}
\renewcommand{\H}{{\mathcal{H}}}
\newcommand{\1}{{\mathbbm{1}}}
\newcommand{\Hom}{{\mathrm{Hom}}}
\newcommand{\Id}{{\mathrm{Id}}}
\newcommand{\simp}[1]{{\Delta^{#1}}}
\newcommand{\osimp}[1]{{\mathring{\Delta}^{#1}}}

\newcounter{punkt}

\theoremstyle{plain}
\newtheorem{twierdzenie}[punkt]{Twierdzenie}
\newtheorem{twierdzeniebd}[punkt]{Twierdzenie (bez dowodu)}
\newtheorem{lemat}[punkt]{Lemat}
\newtheorem{stwierdzenie}[punkt]{Stwierdzenie}
\newtheorem{hipoteza}[punkt]{Hipoteza}

\theoremstyle{definition}
\newtheorem{definicja}[punkt]{Definicja}
\newtheorem{fakt}[punkt]{Fakt}
\newtheorem{wniosek}[punkt]{Wniosek}

\theoremstyle{remark}
\newtheorem{uwaga}[punkt]{Uwaga}
\newtheorem{przyklad}[punkt]{Przykład}
\newtheorem{cwiczenie}[punkt]{Ćwiczenie}



\markright{Piotr Suwara\hfill $\ell_2$-metody w~topologii: 20 stycznia 2014 \hfill}
 
\begin{document}

{\bf Kohomologie de Rhama} zostały przypomniane.
I~struktura różniczkowa na rozmaitości.
I~metryka Riemanna. I
\begin{definicja}[gwiazdka Hodge'a]
	$\ast: \Omega^k(M) \to \Omega^{n-k}(M)$, 
	${\ast \ast = (-1)^{k(n-k)}}$,
	${\alpha \wedge (\ast \alpha) = |\alpha^2| \mr{Vol}}$
\end{definicja}

\begin{definicja}[iloczyn skalarny na $\Omega^p(M)$]
	$(\alpha, \beta) = \int_M \alpha \wedge (\ast \beta)$
\end{definicja}

\begin{definicja}
	$\tilde{X}$ niezwarta rozmaitość Riemanna,
	$\Gamma$ grupa działająca kozwarcie na $\tilde{X}$ przez
	izometrie zachowujące orientację.
	
	Niech $A^{0,p} = A^{0,p}(\tilde{X})$ przestrzeń $L^2$-$p$-form
	na $\tilde{X}$, ${A^{0,p} = \overline{A_0^p}^{\| \cdot \|_0}}$,
	gdzie $A_0^p$ przestrzeń $C^\infty$ $p$-form na $\tilde{X}$
	o~zwartym nośniku,
	$\| \omega \|_0^2 = (\omega, \omega)_0 
	= {\int_{\tilde{X}} \omega \wedge (\ast \omega)}$
\end{definicja}

\begin{stwierdzenie}
	$A^{0,p}$ jest przestrzenią Hilberta z~iloczynem skalarnym
	$(\cdot, \cdot)_0$ i~ma rozkład Hodge'a:
	$$A^{0,p} = \overline{d A_0^{p-1}} \oplus \H^p \oplus 
	\overline{d^\ast A_0^{p+1}},$$
	gdzie $\H^p = \{ \omega: d \omega = d^\ast \omega = 0\}$ 
	to przestrzeń form harmonicznych.
\end{stwierdzenie}

\begin{definicja}[$L^2$-kohomologie de Rhama]
	$H_{dR(2)}^p(\tilde{X}) = \bigslant{\ker d}{\overline{\im d}}$
\end{definicja}

\begin{twierdzenie}[Dodziuk 1977]
	Całkowanie form różniczkowych po sympleksach indukuje
	$\Gamma$-izomorfizm $\int: \H^p(\tilde{X} 
	\to \bar{H}_{(2)}^p(\tilde{X})$.
\end{twierdzenie}

\begin{definicja}
	$A^{k,p} = \{ \omega \in A^{0,p} : (\Id + \Delta)^k \omega
	\in A^{0,p} \}$, gdzie $\Delta = d d^\ast + d^\ast d$
	-- odpowiednik przestrzeni Soboleva.
\end{definicja}

\begin{stwierdzenie}[Atiyah]
	Dla każdego $k \geq 0$, $0 \leq p \leq \dim X$, przestrzeń
	$A^{k,p}$ jest uzupełnieniem $A_0^p$ względem normy
	${\|\omega \|_k^2} = {\| (\Id+\Delta)^k \omega \|_0^2}$,
	$\omega \in A_0^p$, oraz $(\Id+\Delta)^k$ jest samosprzężonym
	operatorem na $A^{0,p}$ z~dziedziną $A^{k,p}$.
\end{stwierdzenie}

\begin{lemat}
	Operator $(\Id + \Delta)^k$ jest izomorfizmem przestrzeni Hilberta
	$A^{k,p}$ i~$A^{0,p}$.
\end{lemat}

\begin{lemat}
	$L^2$-formy harmoniczne na $\tilde{X}$ są zamknięte
	i~kozamknięte,
	
	$\H^p(\tilde{X}) = \{ \omega \in A^{0,p} : \Delta \omega = 0 \}
	= \{ \omega \in A^{0,p}: d \omega = d^\ast \omega = 0 \}$.
	
	Ponadto $Z^p(\tilde{X}) = \ker d$ może być zapisane jako
	
	$Z^p(\tilde{X}) = \H^p(\tilde{X}) \oplus 
	\overline{B^p(\tilde{X})}$,
	
	dlatego $\H^p(\tilde{X})$ jest $\Gamma$-izomorficzne
	z~$\bar{H}^p(\tilde{X}) = \bigslant{Z^p}{\overline{B^p}}$.
\end{lemat}











\end{document}
