\documentclass[12pt]{article}
\usepackage{polski}
\usepackage[utf8]{inputenc}
\usepackage[T1]{fontenc}
\usepackage{amsmath}
\usepackage{amsfonts}
\usepackage{fancyhdr}
\usepackage{lastpage}
\usepackage{multirow}
\usepackage{amssymb}
\usepackage{amsthm}
\usepackage{textcomp}
\usepackage{mathabx}
\usepackage{mathtools}
\usepackage{bbm}
\frenchspacing
\usepackage{fullpage}
\setlength{\headsep}{30pt}
\setlength{\headheight}{12pt}
%\setlength{\voffset}{-30pt}
%\setlength{\textheight}{730pt}
\pagestyle{myheadings}

\usepackage{tikz}
%\usepackage{tikz-cd}
\usetikzlibrary{arrows}

\newcommand{\bigslant}[2]{{\left.\raisebox{.2em}{$#1$}\middle/\raisebox{-.2em}{$#2$}\right.}}
\newcommand{\im}{{\mathrm{im}\,}}

\newcommand{\mf}[1]{{\mathfrak{#1}}}
\newcommand{\mb}[1]{{\mathbb{#1}}}
\newcommand{\mc}[1]{{\mathcal{#1}}}
\newcommand{\mr}[1]{{\mathrm{#1}}}

\newcommand{\R}{{\mathbb{R}}}
\newcommand{\Z}{{\mathbb{Z}}}
\newcommand{\N}{{\mathcal{N}}}
\newcommand{\tr}{{\mathrm{tr}}}
\renewcommand{\H}{{\mathcal{H}}}
\newcommand{\1}{{\mathbbm{1}}}
\newcommand{\Hom}{{\mathrm{Hom}}}
\newcommand{\Id}{{\mathrm{Id}}}
\newcommand{\simp}[1]{{\Delta^{#1}}}
\newcommand{\osimp}[1]{{\mathring{\Delta}^{#1}}}

\newcounter{punkt}

\theoremstyle{plain}
\newtheorem{twierdzenie}[punkt]{Twierdzenie}
\newtheorem{twierdzeniebd}[punkt]{Twierdzenie (bez dowodu)}
\newtheorem{lemat}[punkt]{Lemat}
\newtheorem{stwierdzenie}[punkt]{Stwierdzenie}
\newtheorem{hipoteza}[punkt]{Hipoteza}

\theoremstyle{definition}
\newtheorem{definicja}[punkt]{Definicja}
\newtheorem{fakt}[punkt]{Fakt}
\newtheorem{wniosek}[punkt]{Wniosek}

\theoremstyle{remark}
\newtheorem{uwaga}[punkt]{Uwaga}
\newtheorem{przyklad}[punkt]{Przykład}
\newtheorem{cwiczenie}[punkt]{Ćwiczenie}



\markright{Piotr Suwara\hfill $\ell_2$-metody w~topologii: 2 grudnia 2013\hfill}
 
\begin{document}

\begin{definicja}[torus przekształcenia]
	$X$ skończony $\Delta$-kompleks, 
	$f:X \to X$ homeomorfizm symplicjalny,
	wtedy $T_f = \bigslant{X \times I}{(x,0) \sim (f(x),1)}$.
\end{definicja}

\begin{uwaga}
	$X$ skończony $\Delta$-kompleks, to $T_f$ także.
	
	Co więcej, liczba $i$-sympleksów $T_f$ zależy 
	od liczby $i$-sympleksów oraz $(i-1)$-sympleksów w~$X$.
\end{uwaga}

\begin{stwierdzenie}
	Ta liczba nie zależy od $f$.
\end{stwierdzenie}

\begin{fakt}
	$0 \to G \xrightarrow i \Gamma \xrightarrow q H \to 0$,
	to istnieje ${\varphi:H \to \mr{Aut}(G)}$ takie, że
	${\Gamma \simeq G \rtimes_\varphi H}$ 
	wtedy i~tylko wtedy, gdy istnieje $s:H \to \Gamma$ 
	takie, że $q \circ s = \Id_H$.
\end{fakt}

\begin{przyklad}
	$\Gamma_A = \Z^2 \rtimes_A \Z$, gdzie $A \in SL_2(\Z)$, 
	$|\tr A| > 2$, $\Gamma_A \simeq \pi_1(M_A)$, gdzie
	$M_A$ to $3$-wymiarowa Sol-rozmaitość.
\end{przyklad}

\begin{fakt}
	$\pi_1(T_f) = \pi_1(X) \rtimes_{f^\ast} \Z$ (z~van Kampena).
\end{fakt}

\begin{uwaga}
	$T_f$ lokalnie trywialne rozwłóknienie nad okręgiem z~włóknem $X$.
\end{uwaga}

\begin{twierdzenie}[Luck]
	$X$ skończony kompleks, wówczas $\forall_i \beta_i(T_f) = 0$.
\end{twierdzenie}

\begin{lemat}
	$T_f^n \simeq_{htp} T_{f^n}$
\end{lemat}

{\bf Twierdzenie Lucka o~aproksymacji}

\begin{definicja}
	$G$ jest rezydualnie skończona, 
	jeśli istnieje rodzina podgrup $H_i \leq G$,
	gdzie każda $H_i$ jest skończonego indeksu w~$G$, 
	taka, że $\bigcap H_i = \{ e \}$.
\end{definicja}

\begin{przyklad}
	$H_n = n \Z \leq \Z$
\end{przyklad}

\begin{uwaga}
	Jeśli $G$ skończenie generowana,
	to możemy założyć, że $H_i$ są normalne.
\end{uwaga}

\begin{wniosek}
	Grupa $G$ jest rezydualnie skończona, jesli
	$\forall_{g \in G}$ istnieje homomorfizm
	${\varphi: G \to F}$, gdzie $F$ jest grupą skończoną
	taką, że $\varphi(g) \neq e$.
\end{wniosek}

\begin{przyklad}
	$GL_n(\Z)$, podgrupy $\ker \varphi_p$, gdzie 
	$\varphi_p : GL_n(\Z) \to GL_n(\Z_p)$.
\end{przyklad}

\begin{przyklad}
	$\mb{F}_n$ rezydualnie skończone: $\mb{F}_n 
	\hookrightarrow \mb{F}_n
	\hookrightarrow SL_2(\Z)
	\hookrightarrow GL_n(\Z)$
\end{przyklad}

\begin{twierdzenie}[Malcer]
	Skończenie generowana podgrupa $GL_n(R)$
	dla $R$ pierścienia przemiennego z~$1$
	jest rezydualnie skończona.
\end{twierdzenie}

\begin{przyklad}
	$BS(m,n) = \langle a, b: a b^m a^{-1} = b^n \rangle$ dla $m, n \in \Z$.
\end{przyklad}

\begin{hipoteza}
	Czy każda grupa hiperboliczna jest rezydualnie skończona?
\end{hipoteza}

\begin{twierdzenie}[Lucka o~aproksymacji]
	Niech $X$-skończony kompleks o~grupie podstawowej $\pi_1(X)$
	rezydualnie skończonej. Niech $\ldots \subset \Gamma_{m+1}
	\subset \Gamma_m \subset \ldots \subset \pi_1(X) = \pi$
	będzie zstępującym ciągiem normalnych podgrup skończonego indeksu,
	$\bigcap \Gamma_m = \{ e \}$.
	Niech $p_m : X_m \to X$ będzie skończonym nakryciem
	stowarzyszonym z~$\Gamma_m$.
	Wówczas $$\beta_i(X) 
	= \lim_{m \to \infty} \frac{b_i(X_m)}{[\pi : \Gamma_m]}.$$
\end{twierdzenie}

{\bf Dalej jest dowód.}




\end{document}
