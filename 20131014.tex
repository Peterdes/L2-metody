\documentclass[12pt]{article}
\usepackage{polski}
\usepackage[utf8]{inputenc}
\usepackage[T1]{fontenc}
\usepackage{amsmath}
\usepackage{amsfonts}
\usepackage{fancyhdr}
\usepackage{lastpage}
\usepackage{multirow}
\usepackage{amssymb}
\usepackage{amsthm}
\usepackage{textcomp}
\usepackage{mathabx}
\usepackage{mathtools}
\usepackage{bbm}
\frenchspacing
\usepackage{fullpage}
\setlength{\headsep}{30pt}
\setlength{\headheight}{12pt}
%\setlength{\voffset}{-30pt}
%\setlength{\textheight}{730pt}
\pagestyle{myheadings}

\usepackage{tikz}
%\usepackage{tikz-cd}
\usetikzlibrary{arrows}

\newcommand{\bigslant}[2]{{\left.\raisebox{.2em}{$#1$}\middle/\raisebox{-.2em}{$#2$}\right.}}
\newcommand{\im}{{\mathrm{im}\,}}

\newcommand{\mf}[1]{{\mathfrak{#1}}}
\newcommand{\mb}[1]{{\mathbb{#1}}}
\newcommand{\mc}[1]{{\mathcal{#1}}}
\newcommand{\mr}[1]{{\mathrm{#1}}}

\newcommand{\R}{{\mathbb{R}}}
\newcommand{\Z}{{\mathbb{Z}}}
\newcommand{\N}{{\mathcal{N}}}
\newcommand{\tr}{{\mathrm{tr}}}
\renewcommand{\H}{{\mathcal{H}}}
\newcommand{\1}{{\mathbbm{1}}}
\newcommand{\Hom}{{\mathrm{Hom}}}
\newcommand{\Id}{{\mathrm{Id}}}
\newcommand{\simp}[1]{{\Delta^{#1}}}
\newcommand{\osimp}[1]{{\mathring{\Delta}^{#1}}}

\newcounter{punkt}

\theoremstyle{plain}
\newtheorem{twierdzenie}[punkt]{Twierdzenie}
\newtheorem{twierdzeniebd}[punkt]{Twierdzenie (bez dowodu)}
\newtheorem{lemat}[punkt]{Lemat}
\newtheorem{stwierdzenie}[punkt]{Stwierdzenie}
\newtheorem{hipoteza}[punkt]{Hipoteza}

\theoremstyle{definition}
\newtheorem{definicja}[punkt]{Definicja}
\newtheorem{fakt}[punkt]{Fakt}
\newtheorem{wniosek}[punkt]{Wniosek}

\theoremstyle{remark}
\newtheorem{uwaga}[punkt]{Uwaga}
\newtheorem{przyklad}[punkt]{Przykład}
\newtheorem{cwiczenie}[punkt]{Ćwiczenie}



\markright{Piotr Suwara\hfill $\ell_2$-metody w~topologii: 14 października 2013\hfill}
 
\begin{document}

\begin{definicja}[charakterystyka Eulera]
	$\chi(X) = \sum_i (-1)^i \dim_\R C_i(X)$
\end{definicja}

\begin{definicja}[liczby Bettiego]
	$b_i(X) = \dim_\R \H_i(X)$
\end{definicja}

\begin{wniosek}
	$\chi(X) = \sum_i (-1)^i b_i(X)$
\end{wniosek}

\begin{wniosek}[nierówność Morse'a]
	$X$ skończony $\Delta$-kompleks mający $\alpha_i$
	$i$-wymiarowych sympleksów. Wtedy $\forall_{k \geq 0}$
	${ \alpha_k - \alpha_{k-1} + \alpha_{k-2} + \ldots + (-1)^k \alpha_0
	\geq b_k - b_{k-1} + b_{k-2} + \ldots + (-1)^k b_0}$
\end{wniosek}

{\bf Zmiana notacji}: $C_i(X) = \Delta_i(X)$

\begin{stwierdzenie}
	Rzut ortogonalny $Z_i(X) \to \H_i(X)$ indukuje 
	izomorfizm przestrzeni liniowych
	$H_i(X, \R) = \bigslant{Z_i(X)}{B_i(X)} \to \H_i(X)$.
\end{stwierdzenie}

\begin{wniosek}
	$b_i(X) = \dim_\R H_i(X, \R)$
\end{wniosek}

\begin{definicja}[funktor kowariantny $\H_i$]
	$f : X \to Y$ ciągłe, definiujemy ${f_! : \H_i(X) \to \H_i(Y)}$
	jako ${f_\ast:H_i(X, \R) \to H_i(Y, \R)}$ złożone 
	z~odpowiednimi izomorfizmami $\H_i(X) \simeq H_i(X,\R)$.
\end{definicja}

\begin{stwierdzenie}
	Rzut ortogonalny $Z^i(X) \to \H_i(X)$ indukuje
	izomorfizm przestrzeni liniowych
	$H^i(X, \R) = \bigslant{Z^i(X)}{B^i(X)} \to \H_i(X)$.
\end{stwierdzenie}

\begin{definicja}[funktor kontrawariantny $H^i$]
	Definiujemy $f^!: \H_i(Y) \to \H_i(X)$ jako złożenie
	${f^\ast: H^i(Y, \R) \to H^i(X,\R)}$ z~odpowiednimi
	izomorfizmami $\H_i(X) \simeq H^i(X, \R)$.
\end{definicja}

\begin{wniosek}
	$(f_!)^\ast = f^!$
\end{wniosek}

{\bf Transfer}

$\bar{X} \xrightarrow \pi X$ nakrycie regularne, $X = \bar{X}/G$,
$\bar{X}, X$ skończone $\Delta$-kompleksy ($G$ działa w~sposób wolny
przez permutację sympleksów).

\begin{uwaga}
	$\pi$ indukuje $\pi_i:C_i(\bar{X}) \to C_i(X)$:
	dla $\sigma, \tau$ komórek $X$, 
	$\bar{\sigma}, \bar{\tau}$ komórek $\bar{X}$ takich, że
	$\pi(\bar{\sigma}) = \sigma, \pi(\bar{\tau}) = \tau$
	mamy $\langle \pi_i \bar{\sigma}, \tau \rangle
	= \langle \bar{\sigma}, \sum_{g \in G} g \bar{\tau} \rangle$.
	
	Wobec tego $\pi_i^\ast : C_i(X) \to C_i(\bar{X})$ spełnia
	$\pi_i^\ast \tau = \sum_{g \in G} g \bar{\tau}$.
\end{uwaga}

\begin{definicja}[notacja]
	$N \bar{c} = \sum_{g \in G} g \bar{c}$
\end{definicja}

\begin{wniosek}
	$\pi_i^\ast \circ \pi_i = N_i$
	
	$\pi_i \circ \pi_i^\ast = |G|$
\end{wniosek}

\begin{wniosek}
	$\pi_i^\ast: C_i(X) \to C_i(\bar{X})$ jest włożeniem.
\end{wniosek}

\begin{uwaga}
	Mamy też $\pi_!$ indukowane przez $\pi_i$ 
	oraz $\pi^!$ indukowane przez $\pi_i^\ast$
	i~zachodzi ${\pi^! = (\pi_!)^\ast}$
	oraz $\pi_! \circ \pi^! = |G|$.
\end{uwaga}

\begin{stwierdzenie}
	$\pi_! : \H_i(\bar{X})^G \to \H_i(X)$, 
	$\pi^! : \H_i(X) \to \H_i(\bar{X})^G$ są izomorfizmami.
\end{stwierdzenie}
\pagebreak
{\bf Pierścienie grupowe}

\begin{definicja}[pierścień grupowy]
	$R$ to pierścień z~$1$ (nas interesują: $\Z, \mb{Q}, \R, \mb{C}$),
	pierścień grupowy $R G, R[G]$ to zbiór
	${\{f: G \to R : f(g) \neq 0\text{ dla sk. wielu g}\}}$
	\\ z~działaniami ${(f+f')(g) = f(g) + f'(g)}$,
	${ (f \ast f')(g) = \sum_{h \in G} f(h) f'(h^{-1} g) }$.
\end{definicja}

\begin{uwaga}[konwencja]
	$f = \sum_g f(g) 1_g = \sum_g f(g) g$
\end{uwaga}

\begin{fakt}
	$\1 = 1_e$ to jedynka w~$RG$.
\end{fakt}

\begin{fakt}
	Jeśli $R, G$ przemienne, to $RG$ przemienny.
\end{fakt}

\begin{definicja}
	$\displaystyle \ell_2(G) = \left\{ f: G \to \R \middle| \sum_{x \in G} |f(x)|^2 < \infty \right\}$
	
	Jest to przestrzeń Hilberta z~normą 
	$\| f\| = \sqrt{ \sum_{g \in G} |f(x)|^2 }$
	pochodzącej od iloczynu skalarnego
	$\langle f, f' \rangle = \sum_{g \in G} f(x) f'(x)$.
\end{definicja}

\begin{fakt}
	Istnieje włożenie przestrzeni liniowych 
	$\R G \hookrightarrow \ell_2(G)$.
	
	Co więcej, $\R G$ jest gęstym podzbiorem $\ell_2(G)$.
\end{fakt}

{\bf Reprezentacja (lewa) regularna}

\begin{definicja}[działanie $G$ na $\ell_2(G)$]
	$(g f)(x) = (\lambda_g f) (x) = f(g^{-1} x)$
\end{definicja}

\begin{fakt}
	\begin{itemize}
		\item $\lambda_{gh} = \lambda_g \lambda_h$
		\item $\lambda_e = \Id$
		\item $\lambda_{g^{-1}} = \lambda_g^{-1}$
	\end{itemize}

\end{fakt}

\begin{lemat}
	$\lambda_g$ są unitarne na $\ell_2(G)$.
\end{lemat}

\begin{fakt}
	Reprezentacja $\lambda$ rozszerza się liniowo do
	lewego działania $\R G$ na $\ell_2(G)$.
\end{fakt}

\begin{uwaga}
	$\alpha = \sum r(g) g$, $f \in \ell_2(G)$, 
	wtedy ${ \| \alpha f\| \leq \| f \| \sum_g |r(g)|}$.
\end{uwaga}

\begin{uwaga}
	Podobnie możemy zdefiniować prawą reprezentację regularną $\rho$:
	${(fg)(x) = (\rho_g f) (x) = f(xg)}$
	i~rozszerzyć do prawego działania $\R G$ na $\ell_2(G)$.
\end{uwaga}

Niech $Y$ -- nieskończony $\Delta$-kompleks, na którym $G$ działa wolno
przez permutację sympleksów.

Niech działanie $G$ na $Y$ będzie kozwarte, tj. $X = \bigslant{Y}{G}$
jest skończonym $\Delta$-kompleksem.

\begin{przyklad}
	$Y = \R, G = \Z, X = \bigslant Y G  = S^1$
\end{przyklad}

\begin{przyklad}
	$Y = \R^2, G = \Z^2, X = \bigslant Y G  = T^2$
\end{przyklad}

\begin{przyklad}
	$Y = \text{drzewo}, G = \mb{F}_2, X = \bigslant Y G = \text{ósemka}$
\end{przyklad}

\begin{definicja}[$C_i(Y, G)$]
	$K_i(Y, \R)$ grupa symplicjalnych $i$-łańcuchów $Y$,
	\\ $\Sigma_i$ zbiór $i$-wymiarowych sympleksów w~$Y$,
	\\ $K_i(Y, \R) = \{ f: \Sigma_i \to \R \}$.
	
	Niech $C_i(Y) = C_i(Y, G) = {\ell_2(G) \otimes_G K_i(Y, \R)}
	= { \{f: \Sigma_i \to \ell_2(G) | f(g \sigma) = \lambda_g f(\sigma) \} }$.
\end{definicja}

\begin{definicja}[$C_i$ jako przestrzeń Hilberta]
	Niech $\bar{\tau}_i^\mu$ (${\mu \in \{ 1, \ldots, \alpha_i\}}$,
	$\alpha_i$ to liczba $i$-sympleksów w~$X$) będą
	reprezentantami orbit $i$-sympleksów przy działaniu $G$
	na $Y$. Bazą ortonormalną $C_i(Y)$ jest
	$$\left\{ x \otimes \bar{\tau}_i^\mu 
		\middle| x \in G, \mu \in \{1, \ldots, \alpha_i \} \right\}.$$
		
	Równoważnie, dla $f, f' \in C_i(Y)$
	$$\langle f, f' \rangle =
	\sum_{\mu \in \{1, \ldots, \alpha_i \}}
		\left\langle f(\bar{\tau}_i^\mu), f'(\bar{\tau}_i^\mu) 
				\right\rangle_{\ell_2(G)}$$
\end{definicja}

\begin{uwaga}
	$f \in \ell_2(G)$, to elementy 
	$f \otimes \bar{\tau}_i^\mu \in C_i(Y)$ spełniają
	$\| f \otimes \bar{\tau}_i^\mu\| = \| f \|$.
\end{uwaga}

\begin{wniosek}
	Przekształcenie $(f_1, \ldots, f_{\alpha_i}) 
	\mapsto \sum_{\mu = 1}^{\alpha_i} f_\mu \otimes \bar{\tau}_i^\mu$
	zadaje izometryczny $G$-ekwiwariantny izomorfizm
	$$\left( \bigoplus_1^{\alpha_i} \ell_2(G) \right) \to C_i(Y).$$
\end{wniosek}

{\bf Różniczki}

\begin{definicja}
	$d_i$ dane tymi wzorami, co zawsze (wcześniej: $\partial_i$).
\end{definicja}

\begin{lemat}
	Neich $\varphi: (\R G)^n \to (\R G)^m$ morfizm $\R G$-modułów,
	wtedy operator indukowany
	${\tilde{\varphi} = \Id_{\ell_2(G)} \otimes_{\R G} \varphi
	: (\ell_2(G))^n \to (\ell_2(G))^m}$
	jest ograniczony.
\end{lemat}

\begin{wniosek}
	$d_i: C_i(Y) \to C_{i-1}(Y)$ operator ograniczony 
	w~sensie przestrzeni Hilberta.
\end{wniosek}

\begin{definicja}
	$\delta_{i-1} = d_i^\ast$
	
	$Z_i(Y) = \ker d_i, \quad Z^i(Y) = \ker \delta_i$ 
	domknięte podprzestrzenie liniowe w~$C_i(Y)$.
\end{definicja}

\begin{definicja}
	$\H_i(Y,H) = Z_i(Y) \cap Z^i(Y)$ 
	-- harmoniczne $\ell_2$-kołańcuchy na $Y$
	
	$B_i(Y) = \im d_{i+1}, \quad B^i(Y) = \im \delta_{i-1}$ 
	(brzegi i~kobrzegi)
	
	$\bar{B}_i(Y), \bar{B}^i(Y)$ domknięcia podprzestrzeni
	$B_i(Y), B^i(Y)$ w~normie
\end{definicja}

\begin{stwierdzenie}[$\ell_2$-rozkład H-dR]
	$C_i(Y) = \bar{B}^i \operp Z_i = \bar{B}_i \operp Z^i
	= \bar{B}_i \operp \bar{B}^i \operp \H_i$
\end{stwierdzenie}

\begin{definicja}[laplasjan]
	$\Delta_i = d_{i+1} \delta_i + \delta_{i-1} d_i : C_i(Y) \to C_i(Y)$
\end{definicja}

\begin{stwierdzenie}
	$\H_i(Y, G) = \ker \Delta_i$
\end{stwierdzenie}

\begin{twierdzenie}
	$H_i^G(Y, \ell_2(G)) = \bigslant{Z_i(Y)}{B_i(Y)}$
\end{twierdzenie}

\begin{definicja}[$\ell_2$-homologie zredukowane]
	$\bar{H}_i(Y) = \bigslant{Z_i(Y)}{\bar{B}_i(Y)}$
\end{definicja}

\begin{stwierdzenie}
	Rzut ortogonalny $Z_i \to \bar{H}_i(Y)$ indukuje 
	izomorfizm $\bar{H}_i(Y) \simeq \H_i(Y)$.
\end{stwierdzenie}












\end{document}
