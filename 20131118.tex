\documentclass[12pt]{article}
\usepackage{polski}
\usepackage[utf8]{inputenc}
\usepackage[T1]{fontenc}
\usepackage{amsmath}
\usepackage{amsfonts}
\usepackage{fancyhdr}
\usepackage{lastpage}
\usepackage{multirow}
\usepackage{amssymb}
\usepackage{amsthm}
\usepackage{textcomp}
\usepackage{mathabx}
\usepackage{mathtools}
\usepackage{bbm}
\frenchspacing
\usepackage{fullpage}
\setlength{\headsep}{30pt}
\setlength{\headheight}{12pt}
%\setlength{\voffset}{-30pt}
%\setlength{\textheight}{730pt}
\pagestyle{myheadings}

\usepackage{tikz}
%\usepackage{tikz-cd}
\usetikzlibrary{arrows}

\newcommand{\bigslant}[2]{{\left.\raisebox{.2em}{$#1$}\middle/\raisebox{-.2em}{$#2$}\right.}}
\newcommand{\im}{{\mathrm{im}\,}}

\newcommand{\mf}[1]{{\mathfrak{#1}}}
\newcommand{\mb}[1]{{\mathbb{#1}}}
\newcommand{\mc}[1]{{\mathcal{#1}}}
\newcommand{\mr}[1]{{\mathrm{#1}}}

\newcommand{\R}{{\mathbb{R}}}
\newcommand{\Z}{{\mathbb{Z}}}
\newcommand{\N}{{\mathcal{N}}}
\newcommand{\tr}{{\mathrm{tr}}}
\renewcommand{\H}{{\mathcal{H}}}
\newcommand{\1}{{\mathbbm{1}}}
\newcommand{\Hom}{{\mathrm{Hom}}}
\newcommand{\Id}{{\mathrm{Id}}}
\newcommand{\simp}[1]{{\Delta^{#1}}}
\newcommand{\osimp}[1]{{\mathring{\Delta}^{#1}}}

\newcounter{punkt}

\theoremstyle{plain}
\newtheorem{twierdzenie}[punkt]{Twierdzenie}
\newtheorem{twierdzeniebd}[punkt]{Twierdzenie (bez dowodu)}
\newtheorem{lemat}[punkt]{Lemat}
\newtheorem{stwierdzenie}[punkt]{Stwierdzenie}
\newtheorem{hipoteza}[punkt]{Hipoteza}

\theoremstyle{definition}
\newtheorem{definicja}[punkt]{Definicja}
\newtheorem{fakt}[punkt]{Fakt}
\newtheorem{wniosek}[punkt]{Wniosek}

\theoremstyle{remark}
\newtheorem{uwaga}[punkt]{Uwaga}
\newtheorem{przyklad}[punkt]{Przykład}
\newtheorem{cwiczenie}[punkt]{Ćwiczenie}



\markright{Piotr Suwara\hfill $\ell_2$-metody w~topologii: 18 listopada 2013\hfill}
 
\begin{document}

\begin{stwierdzenie}
	$C_i(\tilde{X}) \simeq (\ell_2(G))^{\alpha_i}$, 
	gdzie $\alpha_i$ to liczba $i$-sympleksów $X$,
	więc ${0 \leq \beta_i(X) \leq \alpha_i}$.
\end{stwierdzenie}

\begin{stwierdzenie}
	$p:\bar{X} \to X$ skończone nakrycie,
	to $\beta_i(\bar{X}) = m\beta_i(X)$.
\end{stwierdzenie}

\begin{wniosek}
	Jeśli $X$ nakrywa siebie nietrywialnie,
	to $\forall_i \beta_i(X) = 0$.
\end{wniosek}

\begin{twierdzenie}
	$\chi(X) = \sum_i (-1)^i \beta_i(X) 
	( = \sum_i(-1)^i b_i(X) = \sum_i (-1)^i \alpha_i)$
\end{twierdzenie}

\begin{wniosek}[nierówność Morse'a]
	$X$ spójny $\Delta$-kompleks ze skończonym $(k+1)$-szkieletem,
	wtedy
	$\alpha_k - \alpha_{k-1} + \ldots + (-1)^k \alpha_0 \geq
	\beta_k - \beta_{k-1} + \ldots + (-1)^k \beta_0$
\end{wniosek}

\begin{twierdzenie}
	Istnieje model $K(G, 1)$ (przestrzeń Eilenberga-Maclane'a),
	który jest w~każdym wymiarze skończonym kompleksem.
\end{twierdzenie}

\begin{definicja}[$\ell_2$-liczby Bettiego grupy $G$]
	 $\beta_i(G) = \beta_i(K(G,1))$
\end{definicja}

\begin{przyklad}
	$K(\Z, 1) = S^1, 
	\quad K(\Z_2, 1)= \R P ^\infty, 
	\quad K(\mb{F}_n, 1) = \bigwedge_1^n S^1,$ \\
	${K(G \times H, 1) = K(G, 1) \times K(H, 1)}$
\end{przyklad}

\begin{definicja}
	$G$ ma typ $F_n$, jeśli istnieje model $K(G, 1)$, który ma skończone szkielety w~wymiarach $i$ dla $i\leq n$.
	
	$G$ ma typ $F_\infty$, jeśli ma typ $F_n$ dla każdego $n$.
\end{definicja}

\begin{wniosek}
	Z~konstrukcji $K(G, 1)$ przez dolepianie komórek wynika, że
	\begin{itemize}
		\item $G$ skończenie generowana, to $G$ ma typ $F_1$,
		\item $G$ skończenie prezentowalna, to $G$ ma typ $F_2$.
	\end{itemize}
\end{wniosek}

\begin{wniosek}
	Dla $G$ typu $F_n$ możemy zdefiniować 
	$\beta_i(G) = \beta_i(K(G, 1)^{(n)})$ dla $i<n$.
	
	W~szczególności $\beta_1(G)$ jest dobrze zdefiniowane,
	gdy $G$ jest skończenie prezentowalna.
\end{wniosek}

{\bf Dualność Poincar\'{e}}

$X$ spójny skończony $\Delta$-kompleks, $\pi_1(X) = G$, 
$K(\tilde{X})$ kompleks łańcuchowy $\tilde{X}$.

\begin{definicja}
	${}^nDK_j(\tilde{X}) = \Hom_G(K_{n-j}(\tilde{X}), \Z[G])$ 
	-- kompleks $G$-modułów, 
	${(xf)(c) = f(c) x^{-1}}$ dla $x \in G$, $f \in {}^nDK_j(\tilde{X})$.
\end{definicja}

\begin{definicja}
	$X$ jest \emph{wirtualnym $PD^n$-kompleksem}, jeśli 
	istnieje $S < G$, $[G:S] < \infty$ taka, że
	$K_\ast(\tilde{X})$ jest łańcuchowo homotopijny
	jako $\Z[S]$-kompleks z~${}^nDK_\ast(\tilde{X})$.
\end{definicja}

\begin{definicja}
	Grupa $G$ jest \emph{wirtualnie $PD^n$}, jeśli 
	istnieje $K(G, 1)$, które jest wirtualnym $PD^n$-kompleksem.
\end{definicja}

\begin{twierdzenie}
	$X$ -- skończony wirtualny $PD^n$-kompleks.
	Wówczas istnieje podgrupa skończonego indeksu $S < \pi_1(X)$
	taka, że $S$-moduły Hilbertowskie $\bar{H}_i(\tilde{X})$
	i~$\bar{H}_{n-i}(\tilde{X})$ są izomorficzne.
	
	W~szczególności $\beta_i(X) = \beta_{n-i}(X)$ dla wszystkich
	$i \leq n$ oraz jeśli $\pi_1(X)$ nieskończona,
	to $\beta_n(X) = \beta_0(X) = 0$.
\end{twierdzenie}

\begin{twierdzenie}
	$X$ to $G$-$\Delta$-kompleks, $Y$ to $H$-$\Delta$-kompleks. \\
	Wtedy ${\beta_j(X \times Y, G \times H)
	= \sum \beta_s(X, G) \beta_{j-s}(Y, H)}$.
\end{twierdzenie}

\begin{przyklad}
	$\Sigma_g$ orientowalna powierzchnia genusu $g>0$,
	$\Sigma_g$ rozmaitość, czyli $PD^2$-kompleks, czyli 
	$\beta_0(\Sigma_g) = \beta_2(\Sigma_g) = 0$,
	ale $\chi(\Sigma_g) = 2 - 2g$, czyli
	$\beta_1(\Sigma_g) = 2g  -2$.
\end{przyklad}

{\bf Coś ciekawego}

\begin{hipoteza}[A]
	$Y$ spójny wolny kozwarty $G$-kompleks, wówczas wszystkie
	$\beta_i(Y, G)$ są wymierne.
\end{hipoteza}

\begin{hipoteza}[B]
	$\phi: \Z[G]^m \to \Z[G]^n$ morfizm $\Z[G]$-modułów,
	$\tilde{\phi}$ indukowany operator ograniczony
	$\ell_2(G)^m \to \ell_2(G)^n$, wtedy
	$\dim_G \ker \tilde{\phi}$ jest wymierny.
\end{hipoteza}

\begin{hipoteza}[o dzielnikach zera]
	$G$ beztorsyjna, wówczas $\mb{Q}[G]$ nie posiada żadnych 
	dzielników zera różnych od $0$.
\end{hipoteza}

\begin{fakt}
	Hipotezy A i~B są równoważne i~implikują hipotezę
	o~dzielnikach zera.
\end{fakt}

\begin{definicja}[średnia na grupie $G$]
	To ciągły funkcjonał $m:\ell_\infty(G) \to \R$ taki, że
	$m(f) \geq 0$ dla $f \geq 0$ oraz $m(1_G) = 1$.
\end{definicja}

\begin{definicja}[średnia niezmiennicza]
	To taka średnia, że $m(f) = m(gf)$ (gdzie ${(gf)(x) = f(g^{-1}x)}$.
\end{definicja}

\begin{definicja}[średniowalność]
	$G$ jest średniowalna, jeśli istnieje średnia niezmiennicza na $G$.
\end{definicja}

\begin{przyklad}
	$|G|<\infty$, $m(f) = \frac{1}{|G|} \sum f(g)$
\end{przyklad}

\begin{przyklad}
	Na $\Z$, $m_n(f) = \frac{1}{2n+1} \sum_{-n \leq x \leq n} f(x)$.
	
	Lub $m = w^\ast\lim_{k \to \infty} m_{n_k}$ (słaba granica).
\end{przyklad}

\begin{twierdzenie}[F\o{}lner]
	$G$ średniowalna, generowana przez skończony zbiór $S$
	wtedy i~tylko wtedy, gdy
	$\forall_{\varepsilon > 0} \exists_{F \subset G\text{ sk.}}$
	taki, że $\forall_{s \in S} \frac{|F \Delta sF |}{|F|} \leq \varepsilon$.
\end{twierdzenie}

\begin{lemat}[Mazura]
	$\alpha_i \to \alpha$ słabo, to istnieje ciąg 
	$\alpha_i' = \sum_j c_{ij} \alpha_i$, $c_{ij} \geq 0$,
	$\sum_j c_{ij} = 1$, taki, że $\alpha_i' \to \alpha$ w~normie.
\end{lemat}

\begin{uwaga}
	$0 \to N \to G \to Q \to 0$ dokładny, $N, Q$ średniowalne, 
	to $G$ średniowalna.
	
	Podgrupa grupy średniowalnej jest średniowalna.
	
	Iloraz grupy średniowalnej jest średniowalny.
\end{uwaga}

\begin{przyklad}
	$\mb{F}_k$ nie jest średniowalna dla $k \geq 2$.
\end{przyklad}

































\end{document}
